% dvips -t letter hw_tomo.dvi -o hw_tomo.ps ; ps2pdf hw_tomo.ps
\documentclass[11pt,titlepage,fleqn]{article}

\usepackage{amsmath}
\usepackage{amssymb}
\usepackage{latexsym}
\usepackage[round]{natbib}
%\usepackage{epsfig}
\usepackage{graphicx}
\usepackage{bm}

\usepackage{url}
\usepackage{color}

%--------------------------------------------------------------
%       SPACING COMMANDS (Latex Companion, p. 52)
%--------------------------------------------------------------

\usepackage{setspace}    % double-space or single-space
\usepackage{xspace}

\renewcommand{\baselinestretch}{1.2}

\textwidth 460pt
\textheight 690pt
\oddsidemargin 0pt
\evensidemargin 0pt

% see Latex Companion, p. 85
\voffset     -50pt
\topmargin     0pt
\headsep      20pt
\headheight   15pt
\headheight    0pt
\footskip     30pt
\hoffset       0pt



% the ~ command keeps [Figure 2] always together,
% so that they are not separated by an end-of-line
\newcommand{\refEq}[1]{Equation~(\ref{#1})}
\newcommand{\refEqii}[2]{Equations~(\ref{#1}) and (\ref{#2})}
\newcommand{\refEqiii}[3]{Equations~(\ref{#1}), (\ref{#2}), and (\ref{#3})}
\newcommand{\refEqiiii}[4]{Equations~(\ref{#1}), (\ref{#2}), (\ref{#3}), and (\ref{#4})}
\newcommand{\refEqab}[2]{Equations (\ref{#1})--(\ref{#2})}
\newcommand{\refeq}[1]{Eq.~\ref{#1}}
\newcommand{\refeqii}[2]{Eqs.~\ref{#1} and \ref{#2}}
\newcommand{\refeqiii}[3]{Eqs.~\ref{#1}, \ref{#2}, and \ref{#3}}
\newcommand{\refeqiiii}[4]{Eqs.~\ref{#1}, \ref{#2}, \ref{#3}, and \ref{#4}}
\newcommand{\refeqab}[2]{Eqs.~\ref{#1}--\ref{#2}}

\newcommand{\refFig}[1]{Figure~\ref{#1}}
\newcommand{\refFigii}[2]{Figures~\ref{#1} and \ref{#2}}
\newcommand{\refFigiii}[3]{Figures~\ref{#1}, \ref{#2}, and \ref{#3}}
\newcommand{\refFigab}[2]{Figures~\ref{#1}--\ref{#2}}
\newcommand{\reffig}[1]{Fig.~\ref{#1}}
\newcommand{\reffigii}[2]{Figs.~\ref{#1} and \ref{#2}}
\newcommand{\reffigiii}[3]{Figs.~\ref{#1}, \ref{#2}, and \ref{#3}}
\newcommand{\reffigab}[2]{Figs.~\ref{#1}--\ref{#2}}

\newcommand{\refTab}[1]{Table~\ref{#1}}
\newcommand{\refTabii}[2]{Tables~\ref{#1} and \ref{#2}}
\newcommand{\refTabiii}[3]{Tables~\ref{#1}, \ref{#2}, and \ref{#3}}
\newcommand{\refTabab}[2]{Tables~\ref{#1}--\ref{#2}}

\newcommand{\refSec}[1]{Section~\ref{#1}}
\newcommand{\refSecii}[2]{Sections~\ref{#1} and \ref{#2}}
\newcommand{\refSeciii}[3]{Sections~\ref{#1}, \ref{#2}, and \ref{#3}}
\newcommand{\refSecab}[2]{Sections~\ref{#1}--\ref{#2}}

\newcommand{\refCha}[1]{Chapter~\ref{#1}}
\newcommand{\refApp}[1]{Appendix~\ref{#1}}

\newcommand{\refpg}[1]{p.~\pageref{#1}}

% spacing commands
\newcommand{\fgap}{\vspace{-14pt}}  % figure gap
\newcommand{\tgap}{\vspace{6pt}}    % table gap
\newcommand{\egap}{\vspace{10pt}}   % equation gap


% COMMANDS FROM JOHN'S ARTICLE
%\newcommand{\bmth}[1]{\mbox{\boldmath $ #1 $}}
%\newcommand{\m}[2]{m_{#1}^{(1)}m_{#2}^{(2)}}
%\newcommand{\sml}[1]{\mbox{\tiny $#1$}}
%\newcommand{\script}{\it}
%\newcommand{\dn}[1]{ {}{_{#1}} }
%\newcommand{\up}[1]{ {}{^{#1}} }

% fractions
\newcommand{\sfrac}[2]{\mbox{$\frac{{#1}}{{#2}\rule[-3pt]{0pt}{3pt}}$}}
\newcommand{\hlf}{\sfrac{1}{2}}
%\newcommand{\twothirds}{\sfrac{2}{3}}   % included in GJI
\newcommand{\fourthirds}{\sfrac{4}{3}}

\newcommand{\tp}{(\theta, \phi)}
\newcommand{\funa}[1]{{#1} \tp}
\newcommand{\funb}[1]{{#1}(\theta, \phi, t)}
\newcommand{\spo}[1]{{#1}_{\rm SP}}
\newcommand{\npo}[1]{{#1}_{\rm NP}}

\newcommand{\alm}{A_{lm}}
\newcommand{\blm}{B_{lm}}
\newcommand{\phom}{\psi_{\rm hom}}
\newcommand{\phet}{\psi_{\rm het}}
\newcommand{\chom}{c_{\rm hom}}
\newcommand{\chet}{c_{\rm het}}
\newcommand{\cmean}{c_{\rm mean}}
\newcommand{\cprem}{c_{\rm PREM}}
\newcommand{\dq}{\partial_{\theta}}
\newcommand{\dqq}{\partial_{\theta \theta}}
\newcommand{\latlon}[2]{$({#1}^\circ,\,{#2}^\circ)$}
\newcommand{\lalo}{(lat,~lon)}
\newcommand{\latf}{\bar{\theta}_f}

\newcommand{\lammin}{\Lambda_{\rm min}}
\newcommand{\lambar}{\bar{\Lambda}}

\newcommand{\dc}{\overline{\delta c}}
\newcommand{\pk}{\phi_k}
\newcommand{\pf}{\phi_f}
\newcommand{\tf}{\theta_f}
\newcommand{\cost}{\cos \theta}
\newcommand{\sint}{\sin \theta}
\newcommand{\plcost}{P_l(\cost)}
\newcommand{\plmt}{P_{lm}(\cost)}
\newcommand{\plmx}{P_{lm}(x)}
\newcommand{\coswlt}{\cos \omega_l t}
\newcommand{\sinwlt}{\sin \omega_l t}
\newcommand{\coswltau}{\cos\,\omega_l \tau}
\newcommand{\sinwltau}{\sin\,\omega_l \tau}
\newcommand{\delnu}{[\nabla^2 u]_{\rm nu}}
\newcommand{\delan}{[\nabla^2 u]_{\rm an}}
\newcommand{\dom}{{\tt dom}}
\newcommand{\abc}{(\alpha, \, \beta, \, \gamma)}
\newcommand{\const}{{\rm const}}
\newcommand{\andi}{\hspace{20pt} {\rm and} \hspace{20pt}}

\newcommand{\tper}[1]{$T$={#1}s}
\newcommand{\lmax}[1]{$lmax$={#1}}
\newcommand{\epsi}[1]{$\epsilon$={#1}}
\newcommand{\q}[1]{$q$={#1}}
\newcommand{\qmax}{q_{\rm max}}
\newcommand{\qmin}{q_{\rm min}}
\newcommand{\eg}{e.g.,\xspace}
\newcommand{\ie}{i.e.,\xspace}

\newcommand{\delmin}{\ensuremath{\Delta_{\rm{min}}}}
\newcommand{\delmax}{\ensuremath{\Delta_{\rm max}}}
\newcommand{\alphamax}{\ensuremath{\alpha_{\rm max}}}
\newcommand{\vd}[2]{\ensuremath{{\mathbf{#1}}_{#2}}}
\newcommand{\vu}[2]{\ensuremath{{\mathbf{#1}}^{#2}}}

%---------------------------------------------------------
% commands from Tromp
%---------------------------------------------------------

%bold lowercase roman letters

\newcommand{\ba}{\mbox{${\bf a}$}}
\newcommand{\bb}{\mbox{${\bf b}$}}
\newcommand{\bc}{\mbox{${\bf c}$}}
\newcommand{\bd}{\mbox{${\bf d}$}}
\newcommand{\be}{\mbox{${\bf e}$}}
\newcommand{\bef}{\mbox{${\bf f}$}}
\newcommand{\bg}{\mbox{${\bf g}$}}
\newcommand{\bh}{\mbox{${\bf h}$}}
\newcommand{\bi}{\mbox{${\bf i}$}}
\newcommand{\bj}{\mbox{${\bf j}$}}
\newcommand{\bk}{\mbox{${\bf k}$}}
\newcommand{\bl}{\mbox{${\bf l}$}}
\newcommand{\bem}{\mbox{${\bf m}$}}
\newcommand{\bn}{\mbox{${\bf n}$}}
\newcommand{\bo}{\mbox{${\bf o}$}}
\newcommand{\bp}{\mbox{${\bf p}$}}
\newcommand{\bq}{\mbox{${\bf q}$}}
\newcommand{\br}{\mbox{${\bf r}$}}
\newcommand{\bs}{\mbox{${\bf s}$}}
\newcommand{\bt}{\mbox{${\bf t}$}}
\newcommand{\bu}{\mbox{${\bf u}$}}
\newcommand{\bv}{\mbox{${\bf v}$}}
\newcommand{\bw}{\mbox{${\bf w}$}}
\newcommand{\bx}{\mbox{${\bf x}$}}
\newcommand{\by}{\mbox{${\bf y}$}}
\newcommand{\bz}{\mbox{${\bf z}$}}

%bold lowercase greek letters

\newcommand{\balpha}{\mbox{\boldmath $\bf \alpha$}}
\newcommand{\bbeta}{\mbox{\boldmath $\bf \beta$}}
\newcommand{\bgamma}{\mbox{\boldmath $\bf \gamma$}}
\newcommand{\bdelta}{\mbox{\boldmath $\bf \delta$}}
\newcommand{\bepsilon}{\mbox{\boldmath $\bf \epsilon$}}
\newcommand{\beps}{\mbox{\boldmath $\bf \varepsilon$}}
\newcommand{\bzeta}{\mbox{\boldmath $\bf \zeta$}}
\newcommand{\boeta}{\mbox{\boldmath $\bf \eta$}}
\newcommand{\btheta}{\mbox{\boldmath $\bf \theta$}}
\newcommand{\bkappa}{\mbox{\boldmath $\bf \kappa$}}
\newcommand{\blambda}{\mbox{\boldmath $\bf \lambda$}}
\newcommand{\bmu}{\mbox{\boldmath $\bf \mu$}}
\newcommand{\bnu}{\mbox{\boldmath $\bf \nu$}}
\newcommand{\bxi}{\mbox{\boldmath $\bf \xi$}}
\newcommand{\bpi}{\mbox{\boldmath $\bf \pi$}}
\newcommand{\bsigma}{\mbox{\boldmath $\bf \sigma$}}
\newcommand{\btau}{\mbox{\boldmath $\bf \tau$}}
\newcommand{\bupsilon}{\mbox{\boldmath $\bf \upsilon$}}
\newcommand{\bphi}{\mbox{\boldmath $\bf \phi$}}
\newcommand{\bchi}{\mbox{\boldmath $\bf \chi$}}
\newcommand{\bpsi}{\mbox{\boldmath$ \bf \psi$}}
\newcommand{\bomega}{\mbox{\boldmath $\bf \omega$}}
\newcommand{\bom}{\mbox{\boldmath $\bf \omega$}}
\newcommand{\brho}{\mbox{\boldmath $\bf \rho$}}

%bold uppercase roman letters

\newcommand{\bA}{\mbox{${\bf A}$}}
\newcommand{\bB}{\mbox{${\bf B}$}}
\newcommand{\bC}{\mbox{${\bf C}$}}
\newcommand{\bD}{\mbox{${\bf D}$}}
\newcommand{\bE}{\mbox{${\bf E}$}}
\newcommand{\bF}{\mbox{${\bf F}$}}
\newcommand{\bG}{\mbox{${\bf G}$}}
\newcommand{\bH}{\mbox{${\bf H}$}}
\newcommand{\bI}{\mbox{${\bf I}$}}
\newcommand{\bJ}{\mbox{${\bf J}$}}
\newcommand{\bK}{\mbox{${\bf K}$}}
\newcommand{\bL}{\mbox{${\bf L}$}}
\newcommand{\bM}{\mbox{${\bf M}$}}
\newcommand{\bN}{\mbox{${\bf N}$}}
\newcommand{\bO}{\mbox{${\bf O}$}}
\newcommand{\bP}{\mbox{${\bf P}$}}
\newcommand{\bQ}{\mbox{${\bf Q}$}}
\newcommand{\bR}{\mbox{${\bf R}$}}
\newcommand{\bS}{\mbox{${\bf S}$}}
\newcommand{\bT}{\mbox{${\bf T}$}}
\newcommand{\bU}{\mbox{${\bf U}$}}
\newcommand{\bV}{\mbox{${\bf V}$}}
\newcommand{\bW}{\mbox{${\bf W}$}}
\newcommand{\bX}{\mbox{${\bf X}$}}
\newcommand{\bY}{\mbox{${\bf Y}$}}
\newcommand{\bZ}{\mbox{${\bf Z}$}}

%bold uppercase greek letters

\newcommand{\bGamma}{\mbox{\boldmath $\bf \Gamma$}}
\newcommand{\bDelta}{\mbox{\boldmath $\bf \Delta$}}
\newcommand{\bLambda}{\mbox{\boldmath $\bf \Lambda$}}
\newcommand{\bXi}{\mbox{\boldmath $\bf \Xi$}}
\newcommand{\bPi}{\mbox{\boldmath $\bf \Pi$}}
\newcommand{\bSigma}{\mbox{\boldmath $\bf \Sigma$}}
\newcommand{\bUpsilon}{\mbox{\boldmath $\bf \Upsilon$}}
\newcommand{\bPhi}{\mbox{\boldmath $\bf \Phi$}}
\newcommand{\bPsi}{\mbox{\boldmath $\bf \Psi$}}
\newcommand{\bOmega}{\mbox{\boldmath $\bf \Omega$}}
\newcommand{\bOm}{\mbox{\boldmath $\bf \Omega$}}


%upper case sans serif letters

\newcommand{\ssA}{\mbox{${\sf A}$}}
\newcommand{\ssB}{\mbox{${\sf B}$}}
\newcommand{\ssC}{\mbox{${\sf C}$}}
\newcommand{\ssD}{\mbox{${\sf D}$}}
\newcommand{\ssE}{\mbox{${\sf E}$}}
\newcommand{\ssF}{\mbox{${\sf F}$}}
\newcommand{\ssG}{\mbox{${\sf G}$}}
\newcommand{\ssH}{\mbox{${\sf H}$}}
\newcommand{\ssI}{\mbox{${\sf I}$}}
\newcommand{\ssJ}{\mbox{${\sf J}$}}
\newcommand{\ssK}{\mbox{${\sf K}$}}
\newcommand{\ssL}{\mbox{${\sf L}$}}
\newcommand{\ssM}{\mbox{${\sf M}$}}
\newcommand{\ssN}{\mbox{${\sf N}$}}
\newcommand{\ssO}{\mbox{${\sf O}$}}
\newcommand{\ssP}{\mbox{${\sf P}$}}
\newcommand{\ssQ}{\mbox{${\sf Q}$}}
\newcommand{\ssR}{\mbox{${\sf R}$}}
\newcommand{\ssS}{\mbox{${\sf S}$}}
\newcommand{\ssT}{\mbox{${\sf T}$}}
\newcommand{\ssU}{\mbox{${\sf U}$}}
\newcommand{\ssV}{\mbox{${\sf V}$}}
\newcommand{\ssW}{\mbox{${\sf W}$}}
\newcommand{\ssX}{\mbox{${\sf X}$}}
\newcommand{\ssY}{\mbox{${\sf Y}$}}
\newcommand{\ssZ}{\mbox{${\sf Z}$}}

%lower case sans serif letters

\newcommand{\ssa}{\mbox{${\sf a}$}}
\newcommand{\ssb}{\mbox{${\sf b}$}}
\newcommand{\ssc}{\mbox{${\sf c}$}}
\newcommand{\ssd}{\mbox{${\sf d}$}}
\newcommand{\sse}{\mbox{${\sf e}$}}
\newcommand{\ssf}{\mbox{${\sf f}$}}
\newcommand{\ssg}{\mbox{${\sf g}$}}
\newcommand{\ssh}{\mbox{${\sf h}$}}
\newcommand{\ssi}{\mbox{${\sf i}$}}
\newcommand{\ssj}{\mbox{${\sf j}$}}
\newcommand{\ssk}{\mbox{${\sf k}$}}
\newcommand{\ssl}{\mbox{${\sf l}$}}
\newcommand{\ssm}{\mbox{${\sf m}$}}
\newcommand{\ssn}{\mbox{${\sf n}$}}
\newcommand{\sso}{\mbox{${\sf o}$}}
%\newcommand{\ssp}{\mbox{${\sf p}$}}    % conflict with some package
\newcommand{\ssq}{\mbox{${\sf q}$}}
\newcommand{\ssr}{\mbox{${\sf r}$}}
\newcommand{\sss}{\mbox{${\sf s}$}}
\newcommand{\sst}{\mbox{${\sf t}$}}
\newcommand{\ssu}{\mbox{${\sf u}$}}
\newcommand{\ssv}{\mbox{${\sf v}$}}
\newcommand{\ssw}{\mbox{${\sf w}$}}
\newcommand{\ssx}{\mbox{${\sf x}$}}
\newcommand{\ssy}{\mbox{${\sf y}$}}
\newcommand{\ssz}{\mbox{${\sf z}$}}

%bold unit vectors

%\newcommand{\bah}{\mbox{$\hat{\mbox{${\bf a}$}}$}}
%\newcommand{\bbh}{\mbox{$\hat{\mbox{${\bf b}$}}$}}
%\newcommand{\bdh}{\mbox{$\hat{\mbox{${\bf d}$}}$}}
%\newcommand{\beh}{\mbox{$\hat{\mbox{${\bf e}$}}$}}
%\newcommand{\bfh}{\mbox{$\hat{\mbox{${\bf f}$}}$}}
%\newcommand{\bgh}{\mbox{$\hat{\mbox{${\bf g}$}}$}}
%\newcommand{\bkh}{\mbox{$\hat{\mbox{${\bf k}$}}$}}
%\newcommand{\bnh}{\mbox{$\hat{\mbox{${\bf n}$}}$}}
%\newcommand{\bph}{\mbox{$\hat{\mbox{${\bf p}$}}$}}
%\newcommand{\brh}{\mbox{$\hat{\mbox{${\bf r}$}}$}}
%\newcommand{\bth}{\mbox{$\hat{\mbox{${\bf t}$}}$}}
%\newcommand{\bxh}{\mbox{$\hat{\mbox{${\bf x}$}}$}}
%\newcommand{\byh}{\mbox{$\hat{\mbox{${\bf y}$}}$}}
%\newcommand{\bzh}{\mbox{$\hat{\mbox{${\bf z}$}}$}}

\newcommand{\bah}{\mbox{\boldmath{$\hat{\rm a}$}}}
\newcommand{\bbh}{\mbox{\boldmath{$\hat{\rm b}$}}}
\newcommand{\bdh}{\mbox{\boldmath{$\hat{\rm d}$}}}
\newcommand{\beh}{\mbox{\boldmath{$\hat{\rm e}$}}}
\newcommand{\bfh}{\mbox{\boldmath{$\hat{\rm f}$}}}
\newcommand{\bgh}{\mbox{\boldmath{$\hat{\rm g}$}}}
\newcommand{\bih}{\mbox{\boldmath{$\hat{\rm i}$}}}
\newcommand{\bjh}{\mbox{\boldmath{$\hat{\rm j}$}}}
\newcommand{\bkh}{\mbox{\boldmath{$\hat{\rm k}$}}}
\newcommand{\bmh}{\mbox{\boldmath{$\hat{\rm m}$}}}
\newcommand{\bnh}{\mbox{\boldmath{$\hat{\rm n}$}}}
\newcommand{\bph}{\mbox{\boldmath{$\hat{\rm p}$}}}
\newcommand{\brh}{\mbox{\boldmath{$\hat{\rm r}$}}}
\newcommand{\bth}{\mbox{\boldmath{$\hat{\rm t}$}}}
\newcommand{\buh}{\mbox{\boldmath{$\hat{\rm u}$}}}
\newcommand{\bxh}{\mbox{\boldmath{$\hat{\rm x}$}}}
\newcommand{\byh}{\mbox{\boldmath{$\hat{\rm y}$}}}
\newcommand{\bzh}{\mbox{\boldmath{$\hat{\rm z}$}}}

\newcommand{\bBh}{\mbox{\boldmath{$\hat{\rm B}$}}}
\newcommand{\bCh}{\mbox{\boldmath{$\hat{\rm C}$}}}
\newcommand{\bFh}{\mbox{\boldmath{$\hat{\rm F}$}}}
\newcommand{\bGh}{\mbox{\boldmath{$\hat{\rm G}$}}}
\newcommand{\bHh}{\mbox{\boldmath{$\hat{\rm H}$}}}
\newcommand{\bLh}{\mbox{\boldmath{$\hat{\rm L}$}}}
\newcommand{\bRh}{\mbox{\boldmath{$\hat{\rm R}$}}}
\newcommand{\bNh}{\mbox{\boldmath{$\hat{\rm N}$}}}
\newcommand{\bSh}{\mbox{\boldmath{$\hat{\rm S}$}}}
\newcommand{\bTh}{\mbox{\boldmath{$\hat{\rm T}$}}}
\newcommand{\bPh}{\mbox{\boldmath{$\hat{\rm P}$}}}

%\newcommand{\bnuh}{\mbox{$\hat{\mbox{\boldmath $\bf \nu$}}$}}
%\newcommand{\bthetah}{\mbox{$\hat{\mbox{\boldmath $\bf \theta$}}$}}
%\newcommand{\betah}{\mbox{$\hat{\mbox{\boldmath $\bf \eta$}}$}}
%\newcommand{\bphih}{\mbox{$\hat{\mbox{\boldmath $\bf \phi$}}$}}
%\newcommand{\bDeltah}{\mbox{$\hat{\mbox{\boldmath $\bf \Delta$}}$}}
%\newcommand{\bThetah}{\mbox{$\hat{\mbox{\boldmath $\bf \Theta$}}$}}
%\newcommand{\bPhih}{\mbox{$\hat{\mbox{\boldmath $\bf \Phi$}}$}}
%\newcommand{\bsigmah}{\mbox{$\hat{\mbox{\boldmath $\bf \sigma$}}$}}

\newcommand{\balphah}{\mbox{\boldmath{$\hat{\alpha}$}}}

\newcommand{\bnuh}{\mbox{\boldmath{$\hat{\nu}$}}}
\newcommand{\bthetah}{\mbox{\boldmath{$\hat{\theta}$}}}
\newcommand{\betah}{\mbox{\boldmath{$\hat{\eta}$}}}
\newcommand{\bphih}{\mbox{\boldmath{$\hat{\phi}$}}}
\newcommand{\blambdah}{\mbox{\boldmath{$\hat{\lambda}$}}}

\newcommand{\bDeltah}{\mbox{\boldmath{$\hat{\Delta}$}}}
\newcommand{\bLambdah}{\mbox{\boldmath{$\hat{\Lambda}$}}}
\newcommand{\bThetah}{\mbox{\boldmath{$\hat{\Theta}$}}}
\newcommand{\bPhih}{\mbox{\boldmath{$\hat{\Phi}$}}}

\newcommand{\bsigmah}{\mbox{\boldmath{$\hat{\sigma}$}}}
\newcommand{\bgammah}{\mbox{\boldmath{$\hat{\gamma}$}}}
\newcommand{\bdeltah}{\mbox{\boldmath{$\hat{\delta}$}}}

\newcommand{\bthetahpr}{\mbox{$\hat{\mbox{\boldmath $\bf \theta$}}
            ^{\raise-.5ex\hbox{$\scriptstyle\prime$}}$}}
\newcommand{\bphihpr}{\mbox{$\hat{\mbox{\boldmath $\bf \phi$}}
            ^{\raise-.5ex\hbox{$\scriptstyle\prime$}}$}}
\newcommand{\bThetahpr}{\mbox{$\hat{\mbox{\boldmath $\bf \Theta$}}
            ^{\raise-.5ex\hbox{$\scriptstyle\prime$}}$}}
\newcommand{\bPhihpr}{\mbox{$\hat{\mbox{\boldmath $\bf \Phi$}}
            ^{\raise-.5ex\hbox{$\scriptstyle\prime$}}$}}

%other miscellaneous definitions (from Tromp)

\newcommand{\om}{\mbox{$\omega$}}
\newcommand{\Om}{\mbox{$\Omega$}}
\newcommand{\ep}{\mbox{$\varepsilon$}}
\newcommand{\p}{\mbox{$\partial$}}
\newcommand{\del}{\mbox{$\nabla$}}
\newcommand{\bdel}{\mbox{\boldmath $\nabla$}}
\newcommand{\bzero}{\mbox{${\bf 0}$}}
\renewcommand{\Re}{\mbox{${\rm R}$}}
\renewcommand{\Im}{\mbox{${\rm I}$}}
\newcommand{\real}{\mbox{$\Re{\rm e}$}}
\newcommand{\imag}{\mbox{$\Im{\rm m}$}}
\newcommand{\sqL}{\mbox{$k$}}
\newcommand{\sqLinv}{\mbox{$k^{-1}$}}
\newcommand{\nunl}{\mbox{${}_n\nu_l$}}
\newcommand{\tdot}{\,.\hspace{-0.98 mm}\raise.6ex\hbox{.}
                   \hspace{-0.98 mm}\raise1.2ex\hbox{.}\,}
\newcommand{\pvint}{\mathbin{\int{\mkern-19.2mu}-}}
\newcommand{\allspace}{\raise.25ex\hbox{$\scriptstyle\bigcirc$}}

%------------------------------------------------------------
% commands from Tromp,Tape,Liu (2005)

\newcommand{\frechet}{Fr\'{e}chet}
\newcommand{\pert}[1]{\delta \ln {#1}}
\newcommand{\pertB}[1]{\frac{\delta {#1}}{{#1}}}

\newcommand{\inter}[2]{{#1}$\sim${#2}$^\dagger$}
                                                                               
\newcommand{\shs}{SH$_{\rm S}$}
\newcommand{\shss}{SH$_{\rm SS}$}
\newcommand{\psvp}{P-SV$_{\rm P}$}
\newcommand{\psvs}{P-SV$_{\rm S}$}
\newcommand{\psvpp}{P-SV$_{\rm PP}$}
\newcommand{\psvss}{P-SV$_{\rm SS}$}
\newcommand{\psvpssp}{P-SV$_{\rm PS+SP}$}
                                                                               
\newcommand{\sdag}{\bs^\dagger}
\newcommand{\fdag}{\bef^\dagger}
\newcommand{\sbdag}{\bar{\bs}^\dagger}
\newcommand{\fbdag}{\bar{\bef}^\dagger}
\newcommand{\Ddag}{\bD^\dagger}
\newcommand{\Dbdag}{\bar{\bD}^\dagger}
                                                                               
%\newcommand{\ka}{K_\alpha}           % \kabr
%\newcommand{\kb}{K_\beta}            % \kbar
%\newcommand{\krp}{K_{\rho'}}         % \krab
%\newcommand{\kk}{K_\kappa}           % \kkmr
%\newcommand{\km}{K_\mu}              % \kmkr
%\newcommand{\kr}{K_\rho}             % \krkm

\newcommand{\kabr}{K_{\alpha(\beta\rho)}}
\newcommand{\kbar}{K_{\beta(\alpha\rho)}}
\newcommand{\krab}{K_{\rho(\alpha\beta)}}
\newcommand{\kkmr}{K_{\kappa(\mu\rho)}}
\newcommand{\kmkr}{K_{\mu(\kappa\rho)}}
\newcommand{\krkm}{K_{\rho(\kappa\mu)}}
\newcommand{\kcbr}{K_{c(\beta\rho)}}
\newcommand{\kbcr}{K_{\beta(c\rho)}}
\newcommand{\krcb}{K_{\rho(c\beta)}}
                                                                               
%\newcommand{\kba}{\bar{K}_\alpha}    % \kbabr
%\newcommand{\kbb}{\bar{K}_\beta}     % \kbbar
%\newcommand{\kbrp}{\bar{K}_{\rho'}}  % \kbrab
%\newcommand{\kbk}{\bar{K}_\kappa}    % \kbkmr
%\newcommand{\kbm}{\bar{K}_\mu}       % \kbmkr
%\newcommand{\kbr}{\bar{K}_\rho}      % \kbrkm

\newcommand{\kbabr}{\bar{K}_{\alpha(\beta\rho)}}
\newcommand{\kbbar}{\bar{K}_{\beta(\alpha\rho)}}
\newcommand{\kbrab}{\bar{K}_{\rho(\alpha\beta)}}
\newcommand{\kbkmr}{\bar{K}_{\kappa(\mu\rho)}}
\newcommand{\kbmkr}{\bar{K}_{\mu(\kappa\rho)}}
\newcommand{\kbrkm}{\bar{K}_{\rho(\kappa\mu)}}
\newcommand{\kbcbr}{\bar{K}_{c(\beta\rho)}}
\newcommand{\kbbcr}{\bar{K}_{\beta(c\rho)}}
\newcommand{\bkrcb}{\bar{K}_{\rho(c\beta)}}

%------------------------------------------------------------
% commands added post-2004

\newcommand{\abr}{$\alpha$-$\beta$-$\rho$}
\newcommand{\cbr}{$c$-$\beta$-$\rho$}
\newcommand{\kmr}{$\kappa$-$\mu$-$\rho$}

% SCEC FILE
\newcommand{\mone}{u_{0x}}
\newcommand{\mtwo}{u_{0y}}

% CONTINUUM TABLES
\newcommand{\delx}{\nabla_x}
\newcommand{\delX}{\nabla_X}
\newcommand{\delxv}{\nabla_x {\bf v}}
\newcommand{\delu}{\nabla {\bf u}}
\newcommand{\detr}[1]{ {\rm det}({#1}) }

% STATS TABLES
\newcommand{\summy}{ \sum_{i=1}^{n} }
\newcommand{\bhi}{\hat{\beta}_0}
\newcommand{\bhs}{\hat{\beta}_1}

\newcommand{\cov}{{\rm cov}}
\newcommand{\var}{{\rm var}}
\newcommand{\erf}{{\rm erf}}
\newcommand{\Var}{{\rm Var}}

\newcommand{\tband}[2]{\mbox{$T = [{#1}\,{\rm s}-{#2}\,{\rm s}]$}}

% earthquake magnitudes
\newcommand{\magB}[1]{$m_{\rm B}\,${#1}}
\newcommand{\magb}[1]{$m_{\rm b}\,${#1}}
\newcommand{\magw}[1]{$M_{\rm w}\,${#1}}
\newcommand{\mags}[1]{$M_{\rm s}\,${#1}}
\newcommand{\magt}[1]{$M_{\rm t}\,${#1}}
\newcommand{\magM}[1]{$M\,${#1}}

\newcommand{\mw}{M_{\rm w}}
\newcommand{\hdur}{\tau_{\rm h}}

% model vector
\newcommand{\smodel}[1]{{\bf m}_{\rm {#1}}}

%------------------------------------------------------------
% from qinya's new commands

% cal font
\newcommand{\cF}{\mbox{$\cal F$}}
\newcommand{\cC}{\mbox{$\cal C$}}

% variations
\newcommand{\dvphi}{\delta\varphi}
\newcommand{\dbS}{\delta\bS}
\newcommand{\dbs}{\delta\bs}
\newcommand{\dbm}{\delta\bem}
\newcommand{\dbc}{\delta\bc}
\newcommand{\dbf}{\delta\bef}
\newcommand{\dm}{\delta m}
\newcommand{\df}{\delta f}
\newcommand{\dF}{\delta F}
\newcommand{\drho}{\delta\rho}

% it font
\newcommand{\iL}{\mathit{L}}
\newcommand{\iF}{\mathit{F}}

\newcommand{\bFz}{\mathbf{F_0}}
\newcommand{\sd}{\dot{s}}
\newcommand{\sdd}{\ddot{s}}

\newcommand{\vp}{V_{\rm P}}
\newcommand{\vs}{V_{\rm S}}
\newcommand{\vb}{V_{\rm B}}

% attenuation
\newcommand{\qp}{Q_{\rm P}}
\newcommand{\qs}{Q_{\rm S}}

% data and synthetics
% amplitude anomalies
\newcommand{\Aamp}{A}
\newcommand{\Aobs}{\Aamp_{\rm obs}}
\newcommand{\Asyn}{\Aamp_{\rm syn}}
% data and synthetics as scalar functions (of time)
\newcommand{\uobs}{u}
\newcommand{\usyn}{s}
% data and synthetics as a discrete vector of points
\newcommand{\uobsb}{\bu}
\newcommand{\usynb}{\bs}

%------------------------------------------------------------
% 2009 PhD thesis

\newcommand{\eq}{\begin{equation}}
\newcommand{\en}{\end{equation}}
\newcommand{\eqa}{\begin{eqnarray}}
\newcommand{\ena}{\end{eqnarray}}

% convention for variables
\newcommand{\misfitvar}{F}
\newcommand{\misfitt}{\misfitvar^{\rm T}}
\newcommand{\misfitw}{\misfitvar^{\rm W}}
%\newcommand{\nparm}{M}           % number of data
%\newcommand{\ndata}{N}           % number of data
\newcommand{\nrec}{R}            % number of stations
\newcommand{\irec}{r}            % staation index
\newcommand{\nsrc}{S}            % number of sources
\newcommand{\isrc}{s}            % source index
\newcommand{\ipick}{p}            % pick index
\newcommand{\ncmp}{C}            % number of components
\newcommand{\normt}{M_{\rm T}}   % normalization for DT adjoint source
\newcommand{\norma}{M_{\rm A}}   % normalization for DA adjoint source
\newcommand{\normtr}[1]{M_{{\rm T}{#1}}}   % normalization for DT adjoint source -- with subscript
\newcommand{\normar}[1]{M_{{\rm A}{#1}}}   % normalization for DA adjoint source -- with subscript

\newcommand{\rmd}{\mathrm{d}}        % roman d for dV in integrals 
\newcommand{\nglob}{{N_{\rm glob}}} 
\newcommand{\mray}{\bem^{\rm ray}}
\newcommand{\mker}{\bem^{\rm ker}}

%------------------------------------------------------------
% subspace paper

\newcommand{\rms}{\mathrm{s}}
\newcommand{\db}{\overline{d}}
\newcommand{\tssT}{\mathsf{T}}
\newcommand{\tssm}{\mathsf{m}}
\newcommand{\tssd}{\mathsf{d}}
\newcommand{\sslambda}{\mathsf{\lambda}}
\newcommand{\ssgamma}{\mathsf{\gamma}}
\newcommand{\ssmu}{\mathsf{\mu}}
\newcommand{\ssbeta}{\mathsf{\beta}}
\newcommand{\ssalpha}{\mathsf{\alpha}}

\newcommand{\ascent}{\gamma}
\newcommand{\bascent}{\bgamma}
\newcommand{\mprior}{\bem_{\rm prior}}
\newcommand{\mpost}{\bem_{\rm post}}
\newcommand{\mtarget}{\bem_{\rm target}}
%\newcommand{\Cprior}{\bC_{\tssm_0}}
\newcommand{\Cds}{\bC_{{\rm D}_s}}

\newcommand{\covd}{\bC_{\rm D}}
\newcommand{\covm}{\bC_{\rm M}}
\newcommand{\covdi}{\bC_{\rm D}^{-1}}
\newcommand{\covmi}{\bC_{\rm M}^{-1}}
\newcommand{\cprior}{\bC_{\rm prior}}
%\newcommand{\Cprior}{\bC_{\rm M}}
\newcommand{\cpriori}{\bC_{\rm prior}^{-1}}
\newcommand{\cpost}{\bC_{\rm post}}
\newcommand{\cposti}{\bC_{\rm post}^{-1}}

\newcommand{\grad}{\hat{\gamma}}
\newcommand{\bgrad}{\bgammah}
\newcommand{\mvec}{\bem}
\newcommand{\hessM}{\bHh}
\newcommand{\hessD}{\bD}
\newcommand{\muvec}{\Delta\bmu}
\newcommand{\Cpost}{\bC_{\rm post}}
%\newcommand{\Cpost}{\bC_{\tssm}}
\newcommand{\Cd}{\bC_{\rm D}}

%\newcommand{\misfitvartilde}{\widetilde{\misfitvar}}
\newcommand{\misfitvartilde}{\misfitvar}
\newcommand{\gradtilde}{\widetilde{\hat{\gamma}}}
%\newcommand{\bgradtilde}{\widetilde{\bgammah}}
\newcommand{\bgradtilde}{\bgammah}
\newcommand{\mvectilde}{\bem}
%\newcommand{\hessMtilde}{\widetilde{\bHh}}
\newcommand{\hessMtilde}{\bHh}
\newcommand{\hessDtilde}{\widetilde{\bD}}
\newcommand{\muvectilde}{\Delta\widetilde{\bmu}}
%\newcommand{\Cposttilde}{\widetilde{\bC}_{\rm post}}
%\newcommand{\Cposttilde}{\widetilde{\bC}_{\tssm}}
\newcommand{\Cposttilde}{\bC_{\rm post}}
\newcommand{\Cdtilde}{\widetilde{\bC}_{\rm D}}
\newcommand{\Btilde}{\widetilde{\bB}}

%------------------------------------------------------------

\newcommand{\frange}[2]{\mbox{{#1}--{#2}~Hz}}
\newcommand{\trange}[2]{\mbox{{#1}--{#2}~s}}
\newcommand{\lrange}[2]{\mbox{$l$={#1}--{#2}}}

% Alessia's paper
%\newcommand{\stga}{Stage~A}
%\newcommand{\stgb}{Stage~B}
%\newcommand{\stgc}{Stage~C}
%\newcommand{\stgd}{Stage~D}
%\newcommand{\stge}{Stage~E}

% names with accents
\newcommand{\vala}{Hj\"orleifsd\'ottir}
\newcommand{\moho}{Mohorovi\v{c}i\'{c}}

% matrix trace
\newcommand{\tra}[1]{\mathrm{tr}\left(#1\right)}

\newcommand{\nn}{\nonumber}

\newcommand{\rref}{\mathrm{rref}}
\newcommand{\proj}{\mathrm{proj}}
\newcommand{\rank}{\mathrm{rank}}
\newcommand{\cond}{\mathrm{cond}}

\newcommand{\fnyq}{f_{\rm Nyq}}

%------------------------------------------------------------
% moment tensor papers by Vipul, Celso, etc

\newcommand{\gd}[2]{$\gamma~{#1}^\circ$, $\delta~{#2}^\circ$}
% XXX probably we want to change sdr to be srd (BUT BE CAREFUL):
\newcommand{\sdr}[3]{strike~${#1}^\circ$, dip~${#2}^\circ$, rake~${#3}^\circ$} 
\newcommand{\mmin}{\ensuremath{M_0}}     % moment tensor global minimum
\newcommand{\mmax}{\ensuremath{M_x}}     % moment tensor global maximum
\newcommand{\vr}{\ensuremath{\mathit{VR}}}
\newcommand{\vrmax}{\ensuremath{\vr_{\max}}}

%------------------------------------------------------------


% modes
\newcommand{\tnl}[2]{\mbox{$_{#1}\ssT_{#2}$}}
\newcommand{\snl}[2]{\mbox{$_{#1}\ssS_{#2}$}}
\newcommand{\snlm}[3]{\mbox{$_{#1}\ssS_{#2}^{#3}$}}

\graphicspath{
  {./figures/}
}

\newcommand{\howmuchtime}{Approximately how much time {\em outside of class and lab time} did you spend on this problem set? Feel free to suggest improvements here.}


\usepackage{pifont}  % \ding

%--------------------------------------------------------------
\begin{document}
%-------------------------------------------------------------

\begin{spacing}{1.2}
\centering
{\large \bf Problem Set 9: Seismic tomography using ray theory} \\
GEOS 626: Applied Seismology, Carl Tape \\
%Assigned: April 4, 2016 --- Due: April 11, 2016 \\
Last compiled: \today
\end{spacing}

%------------------------

\subsection*{Overview}

\begin{itemize}
\item The objective of this assignment is to design a synthetic seismic tomography experiment and to familiarize yourselves with the fundamental aspects of a tomographic inversion:
%
\begin{itemize}
\item model parameterization
\item construction of the matrix of partial derivatives (of each measurement with respect to each model parameter)
\item solving the least-squares problem using regularization
\end{itemize}

\item The source-receiver geometry for this problem is shown in \refFig{fig:geometry}.

\item This example problem can be thought of as a surface-wave tomography problem, whereby the unknown model is a spatial map of surface wave phase speed and the observations are measurements of surface wave traveltimes.

\item The reading and notes on inverse problems from Homework 8 is relevant.
At the end I have listed some additional references.
\nocite{Tape2007,Menke,AsterE2,Tarantola2005}

\item 
%The practical aspects of the assignment are entirely based on Matlab (plus its Mapping Toolbox), and I have provided several scripts that you will need.
% in a reference directory available from the Blackboard website.
The data files and Matlab scripts that you need are in the directory \verb+tomo/+. The two template scripts are \verb+compute_Gik_ray_template.m+ and \verb+tomo_template.m+.

\end{itemize}

%------------------------

\subsection*{Problem 1 (4.0). Constructing the matrix of partial derivatives.}

\begin{enumerate}
\item (0.0) {\em Model parameterization}. A tomographic model is typically described in terms of a perturbation in wave speed structure from some reference model, \ie $\delta\ln c = \delta c/c$.
(Note~\footnote{See the section entitled ``ASIDE: the natural log function used to express tomographic models'' in the math review homework solutions.})
This function may be expanded in terms of a set of $M$ basis functions:
%
\begin{equation}
\delta\ln c(\bx)=\sum_{k=1}^M \delta m_k\,B_k(\bx),
\label{dlnv}
\end{equation}
%
where $\delta m_k$, $k=1,\ldots,M$, represent the model coefficients describing the model perturbation.  Here we choose to use spherical spline basis functions \citep{WangDahlen1995spline,Wang1998}.
\refFig{fig:basis}b shows an example of a spherical spline basis function.

Do the lab exercise in \verb+lab_tomo.pdf+.

%-----------------

\item (1.0) {\em Formulate an entry of the partial derivatives matrix}.
%
From the class notes (\verb+notes_taylor.pdf+), the ray theory traveltime perturbation for the $i$th source-receiver combination is given by
%
\begin{equation}
\delta T_i = -\int_{\mathrm{ray}_i} c^{-1}\,\delta\ln c\;\rmd s,
\label{ray_tomography}
\end{equation}
%
where $\rmd s$ denotes a segment of the $i$th ray.

\begin{enumerate}
\item Assume a homogeneous reference model with $c_0$ everywhere.

Show how \refEq{ray_tomography} leads to 
%
\begin{equation}
\bDelta\bd = \bG\bDelta\bem,
\end{equation}
%
where the data vector is given by
%
\begin{equation}
\bDelta\bd = (\delta T_1,\ldots,\delta T_i,\ldots,\delta T_N)^T,
\label{d}
\end{equation}
%
and the model vector is given by
%
\begin{equation}
\bDelta\bem = (\delta m_1,\ldots,\delta m_k,\ldots,\delta m_M)^T,
\label{m}
\end{equation}

\item Write the expression for the $G_{ik}$ entry of the partial derivatives matrix~$\bG$.
\end{enumerate}

%-----------------

\item (3.0) {\em Compute an entry of the partial derivatives matrix}. \\
Now adapt \verb+compute_Gik_ray_template.m+ to compute the $N \times M$ partial derivatives matrix~$\bG$.
Take into account the following details:
%
\begin{itemize}
\item Assume a homogeneous reference phase speed model with $c_0 = 3500$~m/s.

\item Because the reference model is homogeneous, the ray paths are simply great circles.  Use the built-in Matlab function \verb+gcwaypts+ to compute the ray paths, and use 1000 points or so to effectively describe the ray path (\verb+nump+). The function \verb+distance+ will compute the great-circle distance (in degrees) between two points on a sphere.

\item Each measurement involves a single source--receiver pair. Use the source--receiver ordering provided in \verb+compute_Gik_ray_template.m+ (see also \refFig{fig:geometry}b), which is
%
\begin{verbatim}
i = (isrc-1)*nrec + irec
\end{verbatim}
%
Here index $i$ is the row index (or measurement index), \verb+isrc+ is the source index, and \verb+irec+ is the receiver index.
For example, row $i=3293$ of $\bG$ (and $\bDelta\bd$) corresponds to the pairing of source 25 with receiver 125.

\item Use $M=286$ spherical spline basis functions of order $q=8$; the center-points are in the data file \verb+con_lonlat_q08.dat+.

\item When properly coded, it should take $<$0.1~s to compute a row of $\bG$ and therefore about 3~minutes to compute the entire matrix. Make sure that your code is reasonably fast (though getting the right answer is most important).

\end{itemize}

\begin{enumerate}
\item (0.2) What are the units for $d_i$, $m_k$, and $G_{ik}$?

\item (2.4) Demonstrate that the value of $G_{ik}$ with $i=126$ and $k=204$ is $-10.3747$. \\
(To be safe, check the value \verb+Gik(126,204)+.) \\
Note: Based on the indexing above, the $i=126$ measurement corresponds to the ray path between the \verb+isrc=1+ source and the \verb+irec=126+ receiver.

\item (0.2) What does each row of $\bG$ correspond to?

\item (0.2) What does each column of $\bG$ correspond to?
\end{enumerate}

%-----------------

\item (0.0) Save the partial derivatives matrix
%
\begin{verbatim}
save('Gik','Gik')    % will save the file Gik.mat with the one variable Gik
\end{verbatim}
%
so that you can load it for the next problem.
%
\begin{verbatim}
load Gik             % will load the file Gik.mat with the one variable Gik
\end{verbatim}

\end{enumerate}

%------------------------

%\pagebreak
\subsection*{Problem 2 (4.2). Solving for the unknown wavespeed structure.}

Instructions:
%
\begin{itemize}
\item At this point, you should have saved your partial derivatives matrix, $\bG$. Now modify \verb+tomo_template.m+.

\item In order to do the inverse problem, you will need the data vector of traveltime measurements (\refeq{d}). Load the column vector \verb+measure_vec.dat+ into Matlab, and check that it has dimension $3300 \times 1$.

\item For this problem, we will make the notation change $\bDelta\bem \rightarrow \bem$ and $\bDelta\bd \rightarrow \bd$, in order to simplify the notation.
\end{itemize}

\begin{enumerate}
\item (0.5)
\begin{enumerate}
\item (0.1) Write the symbolic expression for the solution $\bem$ to the least squares problem $\bG\bem=\bd$ in the case where $\bG^T\bG$ is full rank.

\item (0.4) List the dimensions and units of $\bem$, $\bd$, $\bG$, $\bG^T\bG$, and $\bG^T\bd$.
\end{enumerate}

%------------

\item (0.3) Compute $\bem$ using your formula.  What do you get, and why is this the case?

%------------

\item (1.0) If $\bG^T\bG$ is not full rank, then a damping parameter $\lambda$ may be introduced to stabilize (or ``regularize'') the inversion.  One possible solution to the problem is:
%
\begin{equation}
\bem_{\lambda} = \left(\bG^T\bG+\lambda^2\bI\right)^{-1}\bG^T\bd.
\end{equation}
%
\begin{enumerate}

\item (0.6) Compute $\bem_{\lambda}$ for $\lambda = 10$. \\
List the sum of the squares of the entries of the residual vector $\br_\lambda = \bG\bem_\lambda-\bd$; this is the {\em misfit norm} (squared).
List the sum of the squares of the entries of the model vector $\bem_{\lambda}$; this is the {\em model norm} (squared).

%------------

\item (0.0) Compute $\bem_{\lambda}$ for a range of $\lambda$ values: for example, try

\verb+lamvec = 10.^linspace(log10(minlam), log10(maxlam), numlam);+

with $\lambda_{\rm min} = 0.1$ and $\lambda_{\rm max} = 1000$.

%------------

\item (0.4) Plot the misfit norm ($y$-axis) versus model norm ($x$-axis) using log-scaled values.

%The misfit norm is the sum of the squares of the residuals, where the $i$the residual is $(\bG\bem-\bd)_i$. The model norm is the sum of the squares of the model parameters $m_k$.

I recommend transforming the quantities by $\log_{10}$, then using the \verb+plot+ command, rather than dealing with the \verb+loglog+ command.

\end{enumerate}

%------------

\item (1.0) 

\begin{enumerate}
\item (0.6) Plot some of these model vectors by expanding them in the spherical spline basis functions. Use the same color scale for each plot: \verb+caxis([-0.05 0.05])+. (See~Note\footnote{If you want the {\tt seis} color scale from GMT, then type {\tt colormap(seis)} after each figure is done.}.) Superimpose the source locations in each plot.

\item (0.4) Comment on any trends that you see in phase speed maps that you produce. \\
What happens as $\lambda \rightarrow \infty$?
\end{enumerate}

Notes:
%
\begin{itemize}
\item I have pre-computed the necessary matrix, $\bB$, in \verb+tomo_template.m+, such that the function value $\delta\ln c(\phi_j, \theta_j)$ is computed by the dot product $\bB(j,:) \cdot \bem$. This operation is represented by \refEq{dlnv} and can be though of as
%
\begin{equation}
\bc_{\lambda} = \bB \bem_{\lambda},
\end{equation}
%
where $\bc_{\lambda} = \delta\ln c(\phi, \theta)$ is a vector of phase speed perturbations.

\item The simplest way to plot $\bc_{\lambda}$ is probably with the \verb+scatter+ command. But using something like \verb+pcolor+ will produce prettier results.

\end{itemize}

%\pagebreak
\item (1.0)
%
\begin{enumerate}
\item (0.7) Perform an inversion using only the first five sources; this requires using a reduced form of your original partial derivatives matrix. Use the same damping parameters as before. Include plots as before, and discuss your results.

Note: Plot only the five sources that were used in the inversion.

\item (0.3) Repeat using only the first source.
\end{enumerate}

\item (0.4) What are two ways to stabilize the inverse problem based on the experimental design? \\
Hint: How can we construct a matrix $\bG^T\bG$ with fewer zeros?

\end{enumerate}

% %------------------------

\pagebreak
\subsection*{Problem 3 (1.8). Connections with real tomography problems.}

\begin{enumerate}
\item (0.6) Our data vector contains a set of traveltime differences between a wave in our model and a wave in the target model (\ie the unknown Earth model). For real problems, how is this measurement made?

\item (0.6) After the first iteration, we no longer have a uniform ``reference'' model for phase speed. How will we need to adapt our algorithm to accommodate iterations of the phase speed model?

Hint: How will $\bG$, $\bd$, and $\bem$ change?

\item (0.6) In a real problem we might make measurements at several different periods. How can the phase speed for different periods be used to obtain a 3D model of shear wave structure?

Hint: Think about the modes homework.

\end{enumerate}

%------------------------

\subsection*{Problem} \howmuchtime\

%-------------------------------------------------------------
%\pagebreak
\bibliographystyle{agu08}
\bibliography{carl_abbrev,carl_main,carl_him}
%-------------------------------------------------------------

\clearpage\pagebreak

%\vspace{1cm}
\begin{figure}
\hspace{-0.5cm}
\includegraphics[width=17cm]{socal_03b.eps}
\caption[Source--receiver geometry for southern California]
{{
{\em Left}: Example phase speed map and faults for Southern California. The phase speed map is for Rayleigh waves with periods of 20~seconds.
{\em Right}: Source--receiver geometry for the problem in this homework. The \ding{73} symbols denote the locations of 25 earthquakes; the $\circ$ symbols denote the locations of 132 broadband receivers in the Southern California Seismic Network.
\label{fig:geometry}
}}
\end{figure}

\begin{figure}[p]
\hspace{-1.5cm}
\includegraphics[width=19cm]{talk_Gik03_5_ray_hw.eps}
\caption[Basis function]
{{
Example computation for an entry, $G_{ik}$, of the partial derivatives matrix $\bG$, using rays. The row index $i$ is the source-receiver combination, the column index $k$ is the basis function index. The source is denoted by the \ding{73}, the receiver is denoted by the $\triangle$, and the $\circ$ shows the center-point of the spherical spline in (b).
(a)~Ray path for source number~1 and receiver number~126, corresponding to the $i=126$ index of the $N=3300$ ray paths.
(b)~$B_{203}(\bx)$, the spherical spline basis function for index $k=203$. Also shown are the center-points of the $M=286$ spherical splines.
(c)~Spline $B_{203}$ evaluated along the ray path. The value of the phase speed for the reference model is constant, so $G_{ik} = (-1/c) \int_{{\rm ray}_i} B_k\,\rmd s$. In this example, $G_{ik} = -1/(3500\,{\rm m\,s^{-1}})\;(2.45 \times 10^4\,{\rm m})  = -6.98$~s. 
\label{fig:basis}
}}
\end{figure}

%-------------------------------------------------------------
\end{document}
%-------------------------------------------------------------

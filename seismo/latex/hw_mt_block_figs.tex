\begin{figure}
\centering
\includegraphics[width=16cm]{block_figs_new.eps}
\caption[]
{{
Diagram showing notation for vectors and angles for Problem 2.
$\bn$ is the north vector, $\bN$ is the fault normal vector, $\bK$ is the strike vector, and $\bS$ is the slip vector.
The strike angle is $\kappa = 40^\circ$, the dip angle is $\theta = 70^\circ$, and the slip angle is $\sigma = -120^\circ$.
The slip angle is the (signed) angle between the strike vector and the slip vector.
%--------------
% THIS IS IMPORTANT
The sign of $\sigma$ is determined from the direction of $\bN$: rotate $\bK$ to $\bS$ along the shortest angle; if that rotation is counterclockwise, with $\bN$ determining ``up,'' then the angle is positive. (In other words, if $\bK \times \bS$ is in the direction of $\bN$, then $\sigma > 0$; otherwise $\sigma < 0$.)
Another way of explaining it is that you rotate $\bK$ to $\bS$ clockwise (looking down $\bN$); the resultant angle is between $0^\circ$ and $360^\circ$. By convention, the angle is ``wrapped'' to the interval $[-180^\circ, 180^\circ]$, similar to the choice in longitude conventions. In this example $\sigma = 240^\circ = -120^\circ$.
%--------------
Note that the fault vectors $\bN$ and $\bS$ always ``sandwich'' the colored quadrant of the beachball, which contains $\bT = \bu_1$ at its center.
$\bT$ is the $T$-axis, which is the eigenvector associated with positive eigenvalue, $\lambda_1$.
This can be remembered from the fact that $\bTh = (\bNh + \bSh)/\sqrt{2}$.
%Note that the map view of the beachballs shows the upper hemisphere, which differs from the seismological convention of plotting the lower hemisphere.
\label{fig:cmt}
}}
\end{figure}

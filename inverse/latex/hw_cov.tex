% dvips -t letter hw_cov.dvi -o hw_cov.ps ; ps2pdf hw_cov.ps
\documentclass[11pt,titlepage,fleqn]{article}

\usepackage{amsmath}
\usepackage{amssymb}
\usepackage{latexsym}
\usepackage[round]{natbib}
%\usepackage{epsfig}
\usepackage{graphicx}
\usepackage{bm}

\usepackage{url}
\usepackage{color}

%--------------------------------------------------------------
%       SPACING COMMANDS (Latex Companion, p. 52)
%--------------------------------------------------------------

\usepackage{setspace}    % double-space or single-space
\usepackage{xspace}

\renewcommand{\baselinestretch}{1.2}

\textwidth 460pt
\textheight 690pt
\oddsidemargin 0pt
\evensidemargin 0pt

% see Latex Companion, p. 85
\voffset     -50pt
\topmargin     0pt
\headsep      20pt
\headheight   15pt
\headheight    0pt
\footskip     30pt
\hoffset       0pt



% the ~ command keeps [Figure 2] always together,
% so that they are not separated by an end-of-line
\newcommand{\refEq}[1]{Equation~(\ref{#1})}
\newcommand{\refEqii}[2]{Equations~(\ref{#1}) and (\ref{#2})}
\newcommand{\refEqiii}[3]{Equations~(\ref{#1}), (\ref{#2}), and (\ref{#3})}
\newcommand{\refEqiiii}[4]{Equations~(\ref{#1}), (\ref{#2}), (\ref{#3}), and (\ref{#4})}
\newcommand{\refEqab}[2]{Equations (\ref{#1})--(\ref{#2})}
\newcommand{\refeq}[1]{Eq.~\ref{#1}}
\newcommand{\refeqii}[2]{Eqs.~\ref{#1} and \ref{#2}}
\newcommand{\refeqiii}[3]{Eqs.~\ref{#1}, \ref{#2}, and \ref{#3}}
\newcommand{\refeqiiii}[4]{Eqs.~\ref{#1}, \ref{#2}, \ref{#3}, and \ref{#4}}
\newcommand{\refeqab}[2]{Eqs.~\ref{#1}--\ref{#2}}

\newcommand{\refFig}[1]{Figure~\ref{#1}}
\newcommand{\refFigii}[2]{Figures~\ref{#1} and \ref{#2}}
\newcommand{\refFigiii}[3]{Figures~\ref{#1}, \ref{#2}, and \ref{#3}}
\newcommand{\refFigab}[2]{Figures~\ref{#1}--\ref{#2}}
\newcommand{\reffig}[1]{Fig.~\ref{#1}}
\newcommand{\reffigii}[2]{Figs.~\ref{#1} and \ref{#2}}
\newcommand{\reffigiii}[3]{Figs.~\ref{#1}, \ref{#2}, and \ref{#3}}
\newcommand{\reffigab}[2]{Figs.~\ref{#1}--\ref{#2}}

\newcommand{\refTab}[1]{Table~\ref{#1}}
\newcommand{\refTabii}[2]{Tables~\ref{#1} and \ref{#2}}
\newcommand{\refTabiii}[3]{Tables~\ref{#1}, \ref{#2}, and \ref{#3}}
\newcommand{\refTabab}[2]{Tables~\ref{#1}--\ref{#2}}

\newcommand{\refSec}[1]{Section~\ref{#1}}
\newcommand{\refSecii}[2]{Sections~\ref{#1} and \ref{#2}}
\newcommand{\refSeciii}[3]{Sections~\ref{#1}, \ref{#2}, and \ref{#3}}
\newcommand{\refSecab}[2]{Sections~\ref{#1}--\ref{#2}}

\newcommand{\refCha}[1]{Chapter~\ref{#1}}
\newcommand{\refApp}[1]{Appendix~\ref{#1}}

\newcommand{\refpg}[1]{p.~\pageref{#1}}

% spacing commands
\newcommand{\fgap}{\vspace{-14pt}}  % figure gap
\newcommand{\tgap}{\vspace{6pt}}    % table gap
\newcommand{\egap}{\vspace{10pt}}   % equation gap


% COMMANDS FROM JOHN'S ARTICLE
%\newcommand{\bmth}[1]{\mbox{\boldmath $ #1 $}}
%\newcommand{\m}[2]{m_{#1}^{(1)}m_{#2}^{(2)}}
%\newcommand{\sml}[1]{\mbox{\tiny $#1$}}
%\newcommand{\script}{\it}
%\newcommand{\dn}[1]{ {}{_{#1}} }
%\newcommand{\up}[1]{ {}{^{#1}} }

% fractions
\newcommand{\sfrac}[2]{\mbox{$\frac{{#1}}{{#2}\rule[-3pt]{0pt}{3pt}}$}}
\newcommand{\hlf}{\sfrac{1}{2}}
%\newcommand{\twothirds}{\sfrac{2}{3}}   % included in GJI
\newcommand{\fourthirds}{\sfrac{4}{3}}

\newcommand{\tp}{(\theta, \phi)}
\newcommand{\funa}[1]{{#1} \tp}
\newcommand{\funb}[1]{{#1}(\theta, \phi, t)}
\newcommand{\spo}[1]{{#1}_{\rm SP}}
\newcommand{\npo}[1]{{#1}_{\rm NP}}

\newcommand{\alm}{A_{lm}}
\newcommand{\blm}{B_{lm}}
\newcommand{\phom}{\psi_{\rm hom}}
\newcommand{\phet}{\psi_{\rm het}}
\newcommand{\chom}{c_{\rm hom}}
\newcommand{\chet}{c_{\rm het}}
\newcommand{\cmean}{c_{\rm mean}}
\newcommand{\cprem}{c_{\rm PREM}}
\newcommand{\dq}{\partial_{\theta}}
\newcommand{\dqq}{\partial_{\theta \theta}}
\newcommand{\latlon}[2]{$({#1}^\circ,\,{#2}^\circ)$}
\newcommand{\lalo}{(lat,~lon)}
\newcommand{\latf}{\bar{\theta}_f}

\newcommand{\lammin}{\Lambda_{\rm min}}
\newcommand{\lambar}{\bar{\Lambda}}

\newcommand{\dc}{\overline{\delta c}}
\newcommand{\pk}{\phi_k}
\newcommand{\pf}{\phi_f}
\newcommand{\tf}{\theta_f}
\newcommand{\cost}{\cos \theta}
\newcommand{\sint}{\sin \theta}
\newcommand{\plcost}{P_l(\cost)}
\newcommand{\plmt}{P_{lm}(\cost)}
\newcommand{\plmx}{P_{lm}(x)}
\newcommand{\coswlt}{\cos \omega_l t}
\newcommand{\sinwlt}{\sin \omega_l t}
\newcommand{\coswltau}{\cos\,\omega_l \tau}
\newcommand{\sinwltau}{\sin\,\omega_l \tau}
\newcommand{\delnu}{[\nabla^2 u]_{\rm nu}}
\newcommand{\delan}{[\nabla^2 u]_{\rm an}}
\newcommand{\dom}{{\tt dom}}
\newcommand{\abc}{(\alpha, \, \beta, \, \gamma)}
\newcommand{\const}{{\rm const}}
\newcommand{\andi}{\hspace{20pt} {\rm and} \hspace{20pt}}

\newcommand{\tper}[1]{$T$={#1}s}
\newcommand{\lmax}[1]{$lmax$={#1}}
\newcommand{\epsi}[1]{$\epsilon$={#1}}
\newcommand{\q}[1]{$q$={#1}}
\newcommand{\qmax}{q_{\rm max}}
\newcommand{\qmin}{q_{\rm min}}
\newcommand{\eg}{e.g.,\xspace}
\newcommand{\ie}{i.e.,\xspace}

\newcommand{\delmin}{\ensuremath{\Delta_{\rm{min}}}}
\newcommand{\delmax}{\ensuremath{\Delta_{\rm max}}}
\newcommand{\alphamax}{\ensuremath{\alpha_{\rm max}}}
\newcommand{\vd}[2]{\ensuremath{{\mathbf{#1}}_{#2}}}
\newcommand{\vu}[2]{\ensuremath{{\mathbf{#1}}^{#2}}}

%---------------------------------------------------------
% commands from Tromp
%---------------------------------------------------------

%bold lowercase roman letters

\newcommand{\ba}{\mbox{${\bf a}$}}
\newcommand{\bb}{\mbox{${\bf b}$}}
\newcommand{\bc}{\mbox{${\bf c}$}}
\newcommand{\bd}{\mbox{${\bf d}$}}
\newcommand{\be}{\mbox{${\bf e}$}}
\newcommand{\bef}{\mbox{${\bf f}$}}
\newcommand{\bg}{\mbox{${\bf g}$}}
\newcommand{\bh}{\mbox{${\bf h}$}}
\newcommand{\bi}{\mbox{${\bf i}$}}
\newcommand{\bj}{\mbox{${\bf j}$}}
\newcommand{\bk}{\mbox{${\bf k}$}}
\newcommand{\bl}{\mbox{${\bf l}$}}
\newcommand{\bem}{\mbox{${\bf m}$}}
\newcommand{\bn}{\mbox{${\bf n}$}}
\newcommand{\bo}{\mbox{${\bf o}$}}
\newcommand{\bp}{\mbox{${\bf p}$}}
\newcommand{\bq}{\mbox{${\bf q}$}}
\newcommand{\br}{\mbox{${\bf r}$}}
\newcommand{\bs}{\mbox{${\bf s}$}}
\newcommand{\bt}{\mbox{${\bf t}$}}
\newcommand{\bu}{\mbox{${\bf u}$}}
\newcommand{\bv}{\mbox{${\bf v}$}}
\newcommand{\bw}{\mbox{${\bf w}$}}
\newcommand{\bx}{\mbox{${\bf x}$}}
\newcommand{\by}{\mbox{${\bf y}$}}
\newcommand{\bz}{\mbox{${\bf z}$}}

%bold lowercase greek letters

\newcommand{\balpha}{\mbox{\boldmath $\bf \alpha$}}
\newcommand{\bbeta}{\mbox{\boldmath $\bf \beta$}}
\newcommand{\bgamma}{\mbox{\boldmath $\bf \gamma$}}
\newcommand{\bdelta}{\mbox{\boldmath $\bf \delta$}}
\newcommand{\bepsilon}{\mbox{\boldmath $\bf \epsilon$}}
\newcommand{\beps}{\mbox{\boldmath $\bf \varepsilon$}}
\newcommand{\bzeta}{\mbox{\boldmath $\bf \zeta$}}
\newcommand{\boeta}{\mbox{\boldmath $\bf \eta$}}
\newcommand{\btheta}{\mbox{\boldmath $\bf \theta$}}
\newcommand{\bkappa}{\mbox{\boldmath $\bf \kappa$}}
\newcommand{\blambda}{\mbox{\boldmath $\bf \lambda$}}
\newcommand{\bmu}{\mbox{\boldmath $\bf \mu$}}
\newcommand{\bnu}{\mbox{\boldmath $\bf \nu$}}
\newcommand{\bxi}{\mbox{\boldmath $\bf \xi$}}
\newcommand{\bpi}{\mbox{\boldmath $\bf \pi$}}
\newcommand{\bsigma}{\mbox{\boldmath $\bf \sigma$}}
\newcommand{\btau}{\mbox{\boldmath $\bf \tau$}}
\newcommand{\bupsilon}{\mbox{\boldmath $\bf \upsilon$}}
\newcommand{\bphi}{\mbox{\boldmath $\bf \phi$}}
\newcommand{\bchi}{\mbox{\boldmath $\bf \chi$}}
\newcommand{\bpsi}{\mbox{\boldmath$ \bf \psi$}}
\newcommand{\bomega}{\mbox{\boldmath $\bf \omega$}}
\newcommand{\bom}{\mbox{\boldmath $\bf \omega$}}
\newcommand{\brho}{\mbox{\boldmath $\bf \rho$}}

%bold uppercase roman letters

\newcommand{\bA}{\mbox{${\bf A}$}}
\newcommand{\bB}{\mbox{${\bf B}$}}
\newcommand{\bC}{\mbox{${\bf C}$}}
\newcommand{\bD}{\mbox{${\bf D}$}}
\newcommand{\bE}{\mbox{${\bf E}$}}
\newcommand{\bF}{\mbox{${\bf F}$}}
\newcommand{\bG}{\mbox{${\bf G}$}}
\newcommand{\bH}{\mbox{${\bf H}$}}
\newcommand{\bI}{\mbox{${\bf I}$}}
\newcommand{\bJ}{\mbox{${\bf J}$}}
\newcommand{\bK}{\mbox{${\bf K}$}}
\newcommand{\bL}{\mbox{${\bf L}$}}
\newcommand{\bM}{\mbox{${\bf M}$}}
\newcommand{\bN}{\mbox{${\bf N}$}}
\newcommand{\bO}{\mbox{${\bf O}$}}
\newcommand{\bP}{\mbox{${\bf P}$}}
\newcommand{\bQ}{\mbox{${\bf Q}$}}
\newcommand{\bR}{\mbox{${\bf R}$}}
\newcommand{\bS}{\mbox{${\bf S}$}}
\newcommand{\bT}{\mbox{${\bf T}$}}
\newcommand{\bU}{\mbox{${\bf U}$}}
\newcommand{\bV}{\mbox{${\bf V}$}}
\newcommand{\bW}{\mbox{${\bf W}$}}
\newcommand{\bX}{\mbox{${\bf X}$}}
\newcommand{\bY}{\mbox{${\bf Y}$}}
\newcommand{\bZ}{\mbox{${\bf Z}$}}

%bold uppercase greek letters

\newcommand{\bGamma}{\mbox{\boldmath $\bf \Gamma$}}
\newcommand{\bDelta}{\mbox{\boldmath $\bf \Delta$}}
\newcommand{\bLambda}{\mbox{\boldmath $\bf \Lambda$}}
\newcommand{\bXi}{\mbox{\boldmath $\bf \Xi$}}
\newcommand{\bPi}{\mbox{\boldmath $\bf \Pi$}}
\newcommand{\bSigma}{\mbox{\boldmath $\bf \Sigma$}}
\newcommand{\bUpsilon}{\mbox{\boldmath $\bf \Upsilon$}}
\newcommand{\bPhi}{\mbox{\boldmath $\bf \Phi$}}
\newcommand{\bPsi}{\mbox{\boldmath $\bf \Psi$}}
\newcommand{\bOmega}{\mbox{\boldmath $\bf \Omega$}}
\newcommand{\bOm}{\mbox{\boldmath $\bf \Omega$}}


%upper case sans serif letters

\newcommand{\ssA}{\mbox{${\sf A}$}}
\newcommand{\ssB}{\mbox{${\sf B}$}}
\newcommand{\ssC}{\mbox{${\sf C}$}}
\newcommand{\ssD}{\mbox{${\sf D}$}}
\newcommand{\ssE}{\mbox{${\sf E}$}}
\newcommand{\ssF}{\mbox{${\sf F}$}}
\newcommand{\ssG}{\mbox{${\sf G}$}}
\newcommand{\ssH}{\mbox{${\sf H}$}}
\newcommand{\ssI}{\mbox{${\sf I}$}}
\newcommand{\ssJ}{\mbox{${\sf J}$}}
\newcommand{\ssK}{\mbox{${\sf K}$}}
\newcommand{\ssL}{\mbox{${\sf L}$}}
\newcommand{\ssM}{\mbox{${\sf M}$}}
\newcommand{\ssN}{\mbox{${\sf N}$}}
\newcommand{\ssO}{\mbox{${\sf O}$}}
\newcommand{\ssP}{\mbox{${\sf P}$}}
\newcommand{\ssQ}{\mbox{${\sf Q}$}}
\newcommand{\ssR}{\mbox{${\sf R}$}}
\newcommand{\ssS}{\mbox{${\sf S}$}}
\newcommand{\ssT}{\mbox{${\sf T}$}}
\newcommand{\ssU}{\mbox{${\sf U}$}}
\newcommand{\ssV}{\mbox{${\sf V}$}}
\newcommand{\ssW}{\mbox{${\sf W}$}}
\newcommand{\ssX}{\mbox{${\sf X}$}}
\newcommand{\ssY}{\mbox{${\sf Y}$}}
\newcommand{\ssZ}{\mbox{${\sf Z}$}}

%lower case sans serif letters

\newcommand{\ssa}{\mbox{${\sf a}$}}
\newcommand{\ssb}{\mbox{${\sf b}$}}
\newcommand{\ssc}{\mbox{${\sf c}$}}
\newcommand{\ssd}{\mbox{${\sf d}$}}
\newcommand{\sse}{\mbox{${\sf e}$}}
\newcommand{\ssf}{\mbox{${\sf f}$}}
\newcommand{\ssg}{\mbox{${\sf g}$}}
\newcommand{\ssh}{\mbox{${\sf h}$}}
\newcommand{\ssi}{\mbox{${\sf i}$}}
\newcommand{\ssj}{\mbox{${\sf j}$}}
\newcommand{\ssk}{\mbox{${\sf k}$}}
\newcommand{\ssl}{\mbox{${\sf l}$}}
\newcommand{\ssm}{\mbox{${\sf m}$}}
\newcommand{\ssn}{\mbox{${\sf n}$}}
\newcommand{\sso}{\mbox{${\sf o}$}}
%\newcommand{\ssp}{\mbox{${\sf p}$}}    % conflict with some package
\newcommand{\ssq}{\mbox{${\sf q}$}}
\newcommand{\ssr}{\mbox{${\sf r}$}}
\newcommand{\sss}{\mbox{${\sf s}$}}
\newcommand{\sst}{\mbox{${\sf t}$}}
\newcommand{\ssu}{\mbox{${\sf u}$}}
\newcommand{\ssv}{\mbox{${\sf v}$}}
\newcommand{\ssw}{\mbox{${\sf w}$}}
\newcommand{\ssx}{\mbox{${\sf x}$}}
\newcommand{\ssy}{\mbox{${\sf y}$}}
\newcommand{\ssz}{\mbox{${\sf z}$}}

%bold unit vectors

%\newcommand{\bah}{\mbox{$\hat{\mbox{${\bf a}$}}$}}
%\newcommand{\bbh}{\mbox{$\hat{\mbox{${\bf b}$}}$}}
%\newcommand{\bdh}{\mbox{$\hat{\mbox{${\bf d}$}}$}}
%\newcommand{\beh}{\mbox{$\hat{\mbox{${\bf e}$}}$}}
%\newcommand{\bfh}{\mbox{$\hat{\mbox{${\bf f}$}}$}}
%\newcommand{\bgh}{\mbox{$\hat{\mbox{${\bf g}$}}$}}
%\newcommand{\bkh}{\mbox{$\hat{\mbox{${\bf k}$}}$}}
%\newcommand{\bnh}{\mbox{$\hat{\mbox{${\bf n}$}}$}}
%\newcommand{\bph}{\mbox{$\hat{\mbox{${\bf p}$}}$}}
%\newcommand{\brh}{\mbox{$\hat{\mbox{${\bf r}$}}$}}
%\newcommand{\bth}{\mbox{$\hat{\mbox{${\bf t}$}}$}}
%\newcommand{\bxh}{\mbox{$\hat{\mbox{${\bf x}$}}$}}
%\newcommand{\byh}{\mbox{$\hat{\mbox{${\bf y}$}}$}}
%\newcommand{\bzh}{\mbox{$\hat{\mbox{${\bf z}$}}$}}

\newcommand{\bah}{\mbox{\boldmath{$\hat{\rm a}$}}}
\newcommand{\bbh}{\mbox{\boldmath{$\hat{\rm b}$}}}
\newcommand{\bdh}{\mbox{\boldmath{$\hat{\rm d}$}}}
\newcommand{\beh}{\mbox{\boldmath{$\hat{\rm e}$}}}
\newcommand{\bfh}{\mbox{\boldmath{$\hat{\rm f}$}}}
\newcommand{\bgh}{\mbox{\boldmath{$\hat{\rm g}$}}}
\newcommand{\bih}{\mbox{\boldmath{$\hat{\rm i}$}}}
\newcommand{\bjh}{\mbox{\boldmath{$\hat{\rm j}$}}}
\newcommand{\bkh}{\mbox{\boldmath{$\hat{\rm k}$}}}
\newcommand{\bmh}{\mbox{\boldmath{$\hat{\rm m}$}}}
\newcommand{\bnh}{\mbox{\boldmath{$\hat{\rm n}$}}}
\newcommand{\bph}{\mbox{\boldmath{$\hat{\rm p}$}}}
\newcommand{\brh}{\mbox{\boldmath{$\hat{\rm r}$}}}
\newcommand{\bth}{\mbox{\boldmath{$\hat{\rm t}$}}}
\newcommand{\buh}{\mbox{\boldmath{$\hat{\rm u}$}}}
\newcommand{\bxh}{\mbox{\boldmath{$\hat{\rm x}$}}}
\newcommand{\byh}{\mbox{\boldmath{$\hat{\rm y}$}}}
\newcommand{\bzh}{\mbox{\boldmath{$\hat{\rm z}$}}}

\newcommand{\bBh}{\mbox{\boldmath{$\hat{\rm B}$}}}
\newcommand{\bCh}{\mbox{\boldmath{$\hat{\rm C}$}}}
\newcommand{\bFh}{\mbox{\boldmath{$\hat{\rm F}$}}}
\newcommand{\bGh}{\mbox{\boldmath{$\hat{\rm G}$}}}
\newcommand{\bHh}{\mbox{\boldmath{$\hat{\rm H}$}}}
\newcommand{\bLh}{\mbox{\boldmath{$\hat{\rm L}$}}}
\newcommand{\bRh}{\mbox{\boldmath{$\hat{\rm R}$}}}
\newcommand{\bNh}{\mbox{\boldmath{$\hat{\rm N}$}}}
\newcommand{\bSh}{\mbox{\boldmath{$\hat{\rm S}$}}}
\newcommand{\bTh}{\mbox{\boldmath{$\hat{\rm T}$}}}
\newcommand{\bPh}{\mbox{\boldmath{$\hat{\rm P}$}}}

%\newcommand{\bnuh}{\mbox{$\hat{\mbox{\boldmath $\bf \nu$}}$}}
%\newcommand{\bthetah}{\mbox{$\hat{\mbox{\boldmath $\bf \theta$}}$}}
%\newcommand{\betah}{\mbox{$\hat{\mbox{\boldmath $\bf \eta$}}$}}
%\newcommand{\bphih}{\mbox{$\hat{\mbox{\boldmath $\bf \phi$}}$}}
%\newcommand{\bDeltah}{\mbox{$\hat{\mbox{\boldmath $\bf \Delta$}}$}}
%\newcommand{\bThetah}{\mbox{$\hat{\mbox{\boldmath $\bf \Theta$}}$}}
%\newcommand{\bPhih}{\mbox{$\hat{\mbox{\boldmath $\bf \Phi$}}$}}
%\newcommand{\bsigmah}{\mbox{$\hat{\mbox{\boldmath $\bf \sigma$}}$}}

\newcommand{\balphah}{\mbox{\boldmath{$\hat{\alpha}$}}}

\newcommand{\bnuh}{\mbox{\boldmath{$\hat{\nu}$}}}
\newcommand{\bthetah}{\mbox{\boldmath{$\hat{\theta}$}}}
\newcommand{\betah}{\mbox{\boldmath{$\hat{\eta}$}}}
\newcommand{\bphih}{\mbox{\boldmath{$\hat{\phi}$}}}
\newcommand{\blambdah}{\mbox{\boldmath{$\hat{\lambda}$}}}

\newcommand{\bDeltah}{\mbox{\boldmath{$\hat{\Delta}$}}}
\newcommand{\bLambdah}{\mbox{\boldmath{$\hat{\Lambda}$}}}
\newcommand{\bThetah}{\mbox{\boldmath{$\hat{\Theta}$}}}
\newcommand{\bPhih}{\mbox{\boldmath{$\hat{\Phi}$}}}

\newcommand{\bsigmah}{\mbox{\boldmath{$\hat{\sigma}$}}}
\newcommand{\bgammah}{\mbox{\boldmath{$\hat{\gamma}$}}}
\newcommand{\bdeltah}{\mbox{\boldmath{$\hat{\delta}$}}}

\newcommand{\bthetahpr}{\mbox{$\hat{\mbox{\boldmath $\bf \theta$}}
            ^{\raise-.5ex\hbox{$\scriptstyle\prime$}}$}}
\newcommand{\bphihpr}{\mbox{$\hat{\mbox{\boldmath $\bf \phi$}}
            ^{\raise-.5ex\hbox{$\scriptstyle\prime$}}$}}
\newcommand{\bThetahpr}{\mbox{$\hat{\mbox{\boldmath $\bf \Theta$}}
            ^{\raise-.5ex\hbox{$\scriptstyle\prime$}}$}}
\newcommand{\bPhihpr}{\mbox{$\hat{\mbox{\boldmath $\bf \Phi$}}
            ^{\raise-.5ex\hbox{$\scriptstyle\prime$}}$}}

%other miscellaneous definitions (from Tromp)

\newcommand{\om}{\mbox{$\omega$}}
\newcommand{\Om}{\mbox{$\Omega$}}
\newcommand{\ep}{\mbox{$\varepsilon$}}
\newcommand{\p}{\mbox{$\partial$}}
\newcommand{\del}{\mbox{$\nabla$}}
\newcommand{\bdel}{\mbox{\boldmath $\nabla$}}
\newcommand{\bzero}{\mbox{${\bf 0}$}}
\renewcommand{\Re}{\mbox{${\rm R}$}}
\renewcommand{\Im}{\mbox{${\rm I}$}}
\newcommand{\real}{\mbox{$\Re{\rm e}$}}
\newcommand{\imag}{\mbox{$\Im{\rm m}$}}
\newcommand{\sqL}{\mbox{$k$}}
\newcommand{\sqLinv}{\mbox{$k^{-1}$}}
\newcommand{\nunl}{\mbox{${}_n\nu_l$}}
\newcommand{\tdot}{\,.\hspace{-0.98 mm}\raise.6ex\hbox{.}
                   \hspace{-0.98 mm}\raise1.2ex\hbox{.}\,}
\newcommand{\pvint}{\mathbin{\int{\mkern-19.2mu}-}}
\newcommand{\allspace}{\raise.25ex\hbox{$\scriptstyle\bigcirc$}}

%------------------------------------------------------------
% commands from Tromp,Tape,Liu (2005)

\newcommand{\frechet}{Fr\'{e}chet}
\newcommand{\pert}[1]{\delta \ln {#1}}
\newcommand{\pertB}[1]{\frac{\delta {#1}}{{#1}}}

\newcommand{\inter}[2]{{#1}$\sim${#2}$^\dagger$}
                                                                               
\newcommand{\shs}{SH$_{\rm S}$}
\newcommand{\shss}{SH$_{\rm SS}$}
\newcommand{\psvp}{P-SV$_{\rm P}$}
\newcommand{\psvs}{P-SV$_{\rm S}$}
\newcommand{\psvpp}{P-SV$_{\rm PP}$}
\newcommand{\psvss}{P-SV$_{\rm SS}$}
\newcommand{\psvpssp}{P-SV$_{\rm PS+SP}$}
                                                                               
\newcommand{\sdag}{\bs^\dagger}
\newcommand{\fdag}{\bef^\dagger}
\newcommand{\sbdag}{\bar{\bs}^\dagger}
\newcommand{\fbdag}{\bar{\bef}^\dagger}
\newcommand{\Ddag}{\bD^\dagger}
\newcommand{\Dbdag}{\bar{\bD}^\dagger}
                                                                               
%\newcommand{\ka}{K_\alpha}           % \kabr
%\newcommand{\kb}{K_\beta}            % \kbar
%\newcommand{\krp}{K_{\rho'}}         % \krab
%\newcommand{\kk}{K_\kappa}           % \kkmr
%\newcommand{\km}{K_\mu}              % \kmkr
%\newcommand{\kr}{K_\rho}             % \krkm

\newcommand{\kabr}{K_{\alpha(\beta\rho)}}
\newcommand{\kbar}{K_{\beta(\alpha\rho)}}
\newcommand{\krab}{K_{\rho(\alpha\beta)}}
\newcommand{\kkmr}{K_{\kappa(\mu\rho)}}
\newcommand{\kmkr}{K_{\mu(\kappa\rho)}}
\newcommand{\krkm}{K_{\rho(\kappa\mu)}}
\newcommand{\kcbr}{K_{c(\beta\rho)}}
\newcommand{\kbcr}{K_{\beta(c\rho)}}
\newcommand{\krcb}{K_{\rho(c\beta)}}
                                                                               
%\newcommand{\kba}{\bar{K}_\alpha}    % \kbabr
%\newcommand{\kbb}{\bar{K}_\beta}     % \kbbar
%\newcommand{\kbrp}{\bar{K}_{\rho'}}  % \kbrab
%\newcommand{\kbk}{\bar{K}_\kappa}    % \kbkmr
%\newcommand{\kbm}{\bar{K}_\mu}       % \kbmkr
%\newcommand{\kbr}{\bar{K}_\rho}      % \kbrkm

\newcommand{\kbabr}{\bar{K}_{\alpha(\beta\rho)}}
\newcommand{\kbbar}{\bar{K}_{\beta(\alpha\rho)}}
\newcommand{\kbrab}{\bar{K}_{\rho(\alpha\beta)}}
\newcommand{\kbkmr}{\bar{K}_{\kappa(\mu\rho)}}
\newcommand{\kbmkr}{\bar{K}_{\mu(\kappa\rho)}}
\newcommand{\kbrkm}{\bar{K}_{\rho(\kappa\mu)}}
\newcommand{\kbcbr}{\bar{K}_{c(\beta\rho)}}
\newcommand{\kbbcr}{\bar{K}_{\beta(c\rho)}}
\newcommand{\bkrcb}{\bar{K}_{\rho(c\beta)}}

%------------------------------------------------------------
% commands added post-2004

\newcommand{\abr}{$\alpha$-$\beta$-$\rho$}
\newcommand{\cbr}{$c$-$\beta$-$\rho$}
\newcommand{\kmr}{$\kappa$-$\mu$-$\rho$}

% SCEC FILE
\newcommand{\mone}{u_{0x}}
\newcommand{\mtwo}{u_{0y}}

% CONTINUUM TABLES
\newcommand{\delx}{\nabla_x}
\newcommand{\delX}{\nabla_X}
\newcommand{\delxv}{\nabla_x {\bf v}}
\newcommand{\delu}{\nabla {\bf u}}
\newcommand{\detr}[1]{ {\rm det}({#1}) }

% STATS TABLES
\newcommand{\summy}{ \sum_{i=1}^{n} }
\newcommand{\bhi}{\hat{\beta}_0}
\newcommand{\bhs}{\hat{\beta}_1}

\newcommand{\cov}{{\rm cov}}
\newcommand{\var}{{\rm var}}
\newcommand{\erf}{{\rm erf}}
\newcommand{\Var}{{\rm Var}}

\newcommand{\tband}[2]{\mbox{$T = [{#1}\,{\rm s}-{#2}\,{\rm s}]$}}

% earthquake magnitudes
\newcommand{\magB}[1]{$m_{\rm B}\,${#1}}
\newcommand{\magb}[1]{$m_{\rm b}\,${#1}}
\newcommand{\magw}[1]{$M_{\rm w}\,${#1}}
\newcommand{\mags}[1]{$M_{\rm s}\,${#1}}
\newcommand{\magt}[1]{$M_{\rm t}\,${#1}}
\newcommand{\magM}[1]{$M\,${#1}}

\newcommand{\mw}{M_{\rm w}}
\newcommand{\hdur}{\tau_{\rm h}}

% model vector
\newcommand{\smodel}[1]{{\bf m}_{\rm {#1}}}

%------------------------------------------------------------
% from qinya's new commands

% cal font
\newcommand{\cF}{\mbox{$\cal F$}}
\newcommand{\cC}{\mbox{$\cal C$}}

% variations
\newcommand{\dvphi}{\delta\varphi}
\newcommand{\dbS}{\delta\bS}
\newcommand{\dbs}{\delta\bs}
\newcommand{\dbm}{\delta\bem}
\newcommand{\dbc}{\delta\bc}
\newcommand{\dbf}{\delta\bef}
\newcommand{\dm}{\delta m}
\newcommand{\df}{\delta f}
\newcommand{\dF}{\delta F}
\newcommand{\drho}{\delta\rho}

% it font
\newcommand{\iL}{\mathit{L}}
\newcommand{\iF}{\mathit{F}}

\newcommand{\bFz}{\mathbf{F_0}}
\newcommand{\sd}{\dot{s}}
\newcommand{\sdd}{\ddot{s}}

\newcommand{\vp}{V_{\rm P}}
\newcommand{\vs}{V_{\rm S}}
\newcommand{\vb}{V_{\rm B}}

% attenuation
\newcommand{\qp}{Q_{\rm P}}
\newcommand{\qs}{Q_{\rm S}}

% data and synthetics
% amplitude anomalies
\newcommand{\Aamp}{A}
\newcommand{\Aobs}{\Aamp_{\rm obs}}
\newcommand{\Asyn}{\Aamp_{\rm syn}}
% data and synthetics as scalar functions (of time)
\newcommand{\uobs}{u}
\newcommand{\usyn}{s}
% data and synthetics as a discrete vector of points
\newcommand{\uobsb}{\bu}
\newcommand{\usynb}{\bs}

%------------------------------------------------------------
% 2009 PhD thesis

\newcommand{\eq}{\begin{equation}}
\newcommand{\en}{\end{equation}}
\newcommand{\eqa}{\begin{eqnarray}}
\newcommand{\ena}{\end{eqnarray}}

% convention for variables
\newcommand{\misfitvar}{F}
\newcommand{\misfitt}{\misfitvar^{\rm T}}
\newcommand{\misfitw}{\misfitvar^{\rm W}}
%\newcommand{\nparm}{M}           % number of data
%\newcommand{\ndata}{N}           % number of data
\newcommand{\nrec}{R}            % number of stations
\newcommand{\irec}{r}            % staation index
\newcommand{\nsrc}{S}            % number of sources
\newcommand{\isrc}{s}            % source index
\newcommand{\ipick}{p}            % pick index
\newcommand{\ncmp}{C}            % number of components
\newcommand{\normt}{M_{\rm T}}   % normalization for DT adjoint source
\newcommand{\norma}{M_{\rm A}}   % normalization for DA adjoint source
\newcommand{\normtr}[1]{M_{{\rm T}{#1}}}   % normalization for DT adjoint source -- with subscript
\newcommand{\normar}[1]{M_{{\rm A}{#1}}}   % normalization for DA adjoint source -- with subscript

\newcommand{\rmd}{\mathrm{d}}        % roman d for dV in integrals 
\newcommand{\nglob}{{N_{\rm glob}}} 
\newcommand{\mray}{\bem^{\rm ray}}
\newcommand{\mker}{\bem^{\rm ker}}

%------------------------------------------------------------
% subspace paper

\newcommand{\rms}{\mathrm{s}}
\newcommand{\db}{\overline{d}}
\newcommand{\tssT}{\mathsf{T}}
\newcommand{\tssm}{\mathsf{m}}
\newcommand{\tssd}{\mathsf{d}}
\newcommand{\sslambda}{\mathsf{\lambda}}
\newcommand{\ssgamma}{\mathsf{\gamma}}
\newcommand{\ssmu}{\mathsf{\mu}}
\newcommand{\ssbeta}{\mathsf{\beta}}
\newcommand{\ssalpha}{\mathsf{\alpha}}

\newcommand{\ascent}{\gamma}
\newcommand{\bascent}{\bgamma}
\newcommand{\mprior}{\bem_{\rm prior}}
\newcommand{\mpost}{\bem_{\rm post}}
\newcommand{\mtarget}{\bem_{\rm target}}
%\newcommand{\Cprior}{\bC_{\tssm_0}}
\newcommand{\Cds}{\bC_{{\rm D}_s}}

\newcommand{\covd}{\bC_{\rm D}}
\newcommand{\covm}{\bC_{\rm M}}
\newcommand{\covdi}{\bC_{\rm D}^{-1}}
\newcommand{\covmi}{\bC_{\rm M}^{-1}}
\newcommand{\cprior}{\bC_{\rm prior}}
%\newcommand{\Cprior}{\bC_{\rm M}}
\newcommand{\cpriori}{\bC_{\rm prior}^{-1}}
\newcommand{\cpost}{\bC_{\rm post}}
\newcommand{\cposti}{\bC_{\rm post}^{-1}}

\newcommand{\grad}{\hat{\gamma}}
\newcommand{\bgrad}{\bgammah}
\newcommand{\mvec}{\bem}
\newcommand{\hessM}{\bHh}
\newcommand{\hessD}{\bD}
\newcommand{\muvec}{\Delta\bmu}
\newcommand{\Cpost}{\bC_{\rm post}}
%\newcommand{\Cpost}{\bC_{\tssm}}
\newcommand{\Cd}{\bC_{\rm D}}

%\newcommand{\misfitvartilde}{\widetilde{\misfitvar}}
\newcommand{\misfitvartilde}{\misfitvar}
\newcommand{\gradtilde}{\widetilde{\hat{\gamma}}}
%\newcommand{\bgradtilde}{\widetilde{\bgammah}}
\newcommand{\bgradtilde}{\bgammah}
\newcommand{\mvectilde}{\bem}
%\newcommand{\hessMtilde}{\widetilde{\bHh}}
\newcommand{\hessMtilde}{\bHh}
\newcommand{\hessDtilde}{\widetilde{\bD}}
\newcommand{\muvectilde}{\Delta\widetilde{\bmu}}
%\newcommand{\Cposttilde}{\widetilde{\bC}_{\rm post}}
%\newcommand{\Cposttilde}{\widetilde{\bC}_{\tssm}}
\newcommand{\Cposttilde}{\bC_{\rm post}}
\newcommand{\Cdtilde}{\widetilde{\bC}_{\rm D}}
\newcommand{\Btilde}{\widetilde{\bB}}

%------------------------------------------------------------

\newcommand{\frange}[2]{\mbox{{#1}--{#2}~Hz}}
\newcommand{\trange}[2]{\mbox{{#1}--{#2}~s}}
\newcommand{\lrange}[2]{\mbox{$l$={#1}--{#2}}}

% Alessia's paper
%\newcommand{\stga}{Stage~A}
%\newcommand{\stgb}{Stage~B}
%\newcommand{\stgc}{Stage~C}
%\newcommand{\stgd}{Stage~D}
%\newcommand{\stge}{Stage~E}

% names with accents
\newcommand{\vala}{Hj\"orleifsd\'ottir}
\newcommand{\moho}{Mohorovi\v{c}i\'{c}}

% matrix trace
\newcommand{\tra}[1]{\mathrm{tr}\left(#1\right)}

\newcommand{\nn}{\nonumber}

\newcommand{\rref}{\mathrm{rref}}
\newcommand{\proj}{\mathrm{proj}}
\newcommand{\rank}{\mathrm{rank}}
\newcommand{\cond}{\mathrm{cond}}

\newcommand{\fnyq}{f_{\rm Nyq}}

%------------------------------------------------------------
% moment tensor papers by Vipul, Celso, etc

\newcommand{\gd}[2]{$\gamma~{#1}^\circ$, $\delta~{#2}^\circ$}
% XXX probably we want to change sdr to be srd (BUT BE CAREFUL):
\newcommand{\sdr}[3]{strike~${#1}^\circ$, dip~${#2}^\circ$, rake~${#3}^\circ$} 
\newcommand{\mmin}{\ensuremath{M_0}}     % moment tensor global minimum
\newcommand{\mmax}{\ensuremath{M_x}}     % moment tensor global maximum
\newcommand{\vr}{\ensuremath{\mathit{VR}}}
\newcommand{\vrmax}{\ensuremath{\vr_{\max}}}

%------------------------------------------------------------


\graphicspath{
  {./figures/}
}

\newcommand{\howmuchtime}{Approximately how much time {\em outside of class and lab time} did you spend on this problem set? Feel free to suggest improvements here.}


\renewcommand{\baselinestretch}{1.1}

%--------------------------------------------------------------
\begin{document}
%-------------------------------------------------------------

\begin{spacing}{1.2}
\centering
{\large \bf Problem Set 4: Covariance functions and Gaussian random fields} \\
GEOS 627: Inverse Problems and Parameter Estimation, Carl Tape \\
Assigned: February 13, 2017 --- Due: February 20, 2017 \\
Last compiled: \today
\end{spacing}

%------------------------

\subsection*{Overview and instructions}

\begin{enumerate}
\item This problem set deals with three probability distributions: the uniform distribution, the exponential distribution, and the Gaussian distribution.
%These are all examples of {\em generalized Gaussian functions} \citep[][Section 6.6]{Tarantola2005}, which are shown in \refFig{fig:gengauss}.

\item Reading:
\begin{itemize}
\item \citet{Tarantola2005}: Ch.~2 (note Example 2.1) and Sections 5.1, 5.2, 5.3 (note 5.3.3).
\item \citet{AsterE2}: Appendix B
\item class notes
\end{itemize}

\item (\verb+git pull+) \verb+cp covrand_template.m covrand.m+

\end{enumerate}

% \begin{figure}[h]
% \centering
% \includegraphics[width=13cm]{hw_cov_gauss_gen.eps}
% \caption[]
% {{
% Generalized Gaussian functions, as illustrated in Figure 6.6 of \citet{Tarantola2005}.
% These are plots of Eq.~6.68 for $p = 1, 2, 10, 1000$, which include the exponential distribution ($p=1$), the Gaussian distribution ($p=2$), and the uniform distribution ($p \rightarrow \infty$). The parameter $\sigma$ is the $\sigma_p$ of \citet[][eq.~6.53]{Tarantola2005}; it corresponds to the variance only when $p=2$.
% \label{fig:gengauss}
% }}
% \end{figure}

%------------------------

%\pagebreak
\subsection*{Problem 1 (1.0). Gaussian and exponential PDFs}

% The exponential and Gaussian probability density functions are given by \citep[\eg][Appendix~B]{AsterE2}
% %
% \begin{eqnarray}
% f_1(x) &=& \frac{1}{\sigma\sqrt{2}} \exp\left(-\frac{\sqrt{2}|x - \mu|}{\sigma} \right)
% \\
% f_2(x) &=& \frac{1}{\sigma\sqrt{2\pi}} \exp\left(-\frac{(x-\mu)^2}{2\sigma^2} \right)
% \end{eqnarray}

\begin{enumerate}

% XXX MAKE THIS PROBLEM WORTH MORE POINTS IN THE FIGURE XXX
\item (0.5) Consider the {\bf standard} normal distribution, $f_N(x)$.
\begin{enumerate}
\item (0.1) What is the exact expression and approximate value of $f_N(\pm\sigma)$?
\item (0.1) What is the exact expression and approximate value of $f_N(\mu)$?
\item (0.1) Make a plot of $f_N(x)$.
\item (0.2) Label $\sigma$, $\mu$, and $f_N(\pm\sigma)$ on your plot, and plot the three points $(\mu,f_N(\mu))$, $(-\sigma,f_N(-\sigma))$, and $(\sigma,f_N(\sigma))$.
\end{enumerate}

%------------

\item (0.2) Consider the exponential probability density function
%
\begin{eqnarray}
f(x) &=& k\,\exp\left(-\frac{\sqrt{2}|x - \mu|}{\sigma} \right)
\end{eqnarray}
%
Show that $k = 1/(\sigma\sqrt{2})$.

Hint: Split the integration into two intervals in order to eliminate the absolute values.

%------------

\item (0.3) Consider the two Gaussian probability density functions
%
\begin{eqnarray}
f_X(x) &=& k_x \exp\left(-\frac{(x-\mu_x)^2}{2\sigma_x^2} \right)
\\
f_Y(y) &=& k_y \exp\left(-\frac{(y-\mu_y)^2}{2\sigma_y^2} \right)
\end{eqnarray}
%
\begin{enumerate}
\item (0.1) Assuming that the variables $X$ and $Y$ are independent, what is the joint probability density function $f(x,y)$?
\item (0.2) Assuming that the mean is zero and that $f(x,y)$ has circular level surfaces, show that the normalization factor for $f(x,y)$ is $h = 1/(2\pi\sigma^2)$ (such that $f(x,y) = h\,g(x,y)$).
\end{enumerate}

Hints:
\begin{itemize}
\item \citet[][eq. B.28]{AsterE2}
\item What does ``mean zero'' and ``circular Gaussian'' imply about $f(x,y)$?
\item Try the integration using polar coordinates (it is clean).
\end{itemize}

%------------------

\end{enumerate}

%------------------------

%\pagebreak
\subsection*{Problem 2 (2.5). Uniform PDF (and central limit theorem)}

The formulas for expected value and variance are given by
%
\begin{eqnarray}
E[X] &=& \int_{-\infty}^{\infty} x\;f_X(x) \,dx
\label{EX}
\\
\Var[X] &=& E[X^2] - (E[X])^2
\label{var}
\end{eqnarray}
%
where $f_X(x)$ is a probability density function. The expectated value of $g(X)$ is given by
%
\begin{eqnarray}
E[g(X)] &=& \int_{-\infty}^{\infty} g(x)\;f_X(x) \,dx
\label{Egx}
\end{eqnarray}

%=============================

\begin{enumerate}
\item (0.2) Write the expression for a uniform distribution, $f_U(x)$, on the interval $[a,b]$. \\
Write the Matlab\footnote{or whatever computing language you are using} command to generate $n$ samples of $f_U(x)$.
\label{fu}

%------------------

\item (1.0) Using \refEqab{EX}{Egx} (with your $f_U(x)$ in place of $f_X(x)$), show that the expected value and variance for $f_U(x)$ are given by
%
\begin{eqnarray}
E[X] &=& \frac{a+b}{2}
\\
\Var[X] &=& \frac{(b-a)^2}{12}
\end{eqnarray}
%
Hint: You will probably need to use polynomial long division.

%------------------

\pagebreak
\item (0.0) In Matlab, generate $10^5$ or so samples of $f_U(x)$ (remember: a sample of $f_U(x)$ will be a random number between $a$ and $b$), and check that the mean ($\mu$) and variance ($\sigma^2$) of the samples are close to the theoretical values, \ie $\mu \approx E[X]$ and $\sigma^2 \approx \Var[X]$. For the sake of comparison, use
%
\begin{equation*}
a = -\sqrt{12},
\hspace{1cm}
b = 5\sqrt{12}
\end{equation*}
%\begin{eqnarray*}
%a &=& -\sqrt{12}
%\\
%b &=& 5\sqrt{12}
%\end{eqnarray*}
%
{\bf Plot a histogram of your samples} to check that the distribution is flat over the appropriate interval. (No need to turn in this plot.)

%------------------

\item (1.3) The {\bf central limit theorem} is stated in \citet[][Section B.6]{AsterE2}:
%
\begin{quote}
Let $X_1$, $X_2$, \ldots, $X_n$ be independent and identically distributed (IID) random variables with a finite expected value $\mu$ and variance $\sigma^2$. Let
%
\begin{equation}
Z_n = \frac{X_1 + X_2 + \cdots + X_n - n\mu}{\sigma\sqrt{n}}.
\label{Zn}
\end{equation}
%
In the limit as $n$ approaches infinity, the distribution of $Z_n$ approaches the standard normal distribution.
\end{quote}

The central limit theorem works for any kind of distribution. You will demonstrate it using the uniform distribution, $f_U(x)$ (Problem 2-\ref{fu}), for which you know $\mu$ and $\sigma$.

\begin{enumerate}

\item (0.1)
\begin{itemize}
\item Write the expression for $Z_1$.
\item What are the minimum and maximum values of $Z_1$?
\end{itemize}

Note: Your answer should not have $\mu$ or $\sigma$ in the expressions, since these can be written in terms of $a$ and $b$.

\item (0.4)
\begin{itemize}
\item Write the expression for $Z_2$.
\item What are the minimum and maximum possible values of the sum $X_1 + X_2$?
\item What are the minimum and maximum values of $Z_2$?
\end{itemize}

Note: Your answer should not have $\mu$ or $\sigma$ in the expressions, since these can be written in terms of $a$ and $b$.

\item (0.8) By generating samples ($X$) from your $f_U(x)$, demonstrate the central limit theorem by showing a set of histograms of $Z_1$, $Z_2$, $Z_3$, and $Z_{10}$. To obtain each distribution of $Z_n$ (\refeq{Zn}), you will need to repeat the experiment $p$ times; try $p = 10^5$. Center your histograms between $\pm 4$.

Hint:
\begin{itemize}
\item Consider the case of $n=2$. The first ``experiment'' will involve generating two random samples, $X_1$ and $X_2$, of $f_U(x)$. You can then compute $Z_2$ using \refEq{Zn}. You then repeat this process $p$ times and plot a histogram of the $p$ values of $Z_2$.
\end{itemize}

\end{enumerate}

\end{enumerate}

%------------------------

\pagebreak
\subsection*{Problem 3 (3.5). Estimating a covariance matrix from a set of samples}

See the template script \verb+covrand.m+. Let $P$ be the number of samples and $M$ be the number of model parameters describing a single sample. The $i$th sample is represented by the $M \times 1$ vector~$\bem_i$.

It may (or may not) help to attach some physical meaning to these samples. Think of each sample as the functional variation in a single dimension. The set of samples might represent, for example:
%
\begin{itemize}
\item the variation in topography along different transects.

\item the variation in height of an interval of an oscillating wire: each profile represents a different time.

\item the variation of vertical ground displacement with time, as captured by a seismogram: each profile is for a different earthquake.

\end{itemize}
%
In this problem, the goal is to compute a sample covariance matrix, $\bC_P$, and to use it to estimate the covariance function, $C(d)$, that characterizes the samples.

%--------------

\begin{enumerate}

\item (0.0) Run \verb+covrand.m+. Identify the key variables (and their dimensions) that are loaded into Matlab, then comment the \verb+break+ statement and proceed.

% XXX MAKE THIS WORTH MORE POINTS XXX
\item (0.3) Plot 8 samples in a $4 \times 2$ subplot figure, with one sample per subplot and with the same axis scale for each subplot (use \verb+ax0+ from \verb+covrand.m+). Plot each sample using the spatial discretization given by \verb+x+. Either use a default plotting style or \verb+'-.'+(but not \verb+'.'+).

{\em For all $M$-dimensional vectors in the rest of the problem, plot them using the same $y$-axis range and with the spatial discretization given by \verb+x+.}

\label{samps}

%--------------

\item (1.5) Use the first $P=10$ samples to do the following:
%
\begin{enumerate}
\item (0.3) Compute and plot the mean, $\bmu_{10}$.
\item (0.7) Compute and plot the covariance matrix, $\bC_{10}$ (use \verb+imagesc+). Show your code to compute $\bC_{10}$, and do not use the black-box \verb+cov+ function\footnote{If you use {\tt cov} to check, you may need to transpose your matrix of samples to ensure that the resultant matrix is $M \times M$.}.
\item (0.5) Make a scatterplot (use \verb+plot+) of $(\bC_{10})_{kk'}$ versus $D_{kk'} = |x_k - x_{k'}|$, where $\bD$ is provided in \verb+covrand.m+. 

Hint: Try \verb+plot(D,Csamp,'b.');+ where \verb+Csamp+ represents $\bC_{10}$.
\end{enumerate}

%--------------

\item (0.5) Repeat the previous (include plots), but use all 1000 samples. How does the estimated mean and covariance change with increasing the number of samples?

\label{prob:Cd}

%--------------

\item (0.2) Examine the script \verb+covC.m+.
%, which will be needed in Problems 3-\ref{prob:covC} and 4.
Some example plots using \verb+covC.m+ are shown in \refFig{fig:covC2}. Two of the functions plotted are
%
\begin{eqnarray}
C_{\rm gaus}(d) &=&
%\sigma^2 \exp \left( - \frac{ d^2}{2L^2} \right) =
\sigma^2 \exp \left( - \frac{2 d^2}{L'^2} \right) 
\label{Cgaus1}
\\
C_{\rm exp}(d) &=&
%\sigma^2 \exp \left( - \frac{d}{L} \right) =
\sigma^2 \exp \left( - \frac{2 d}{L'} \right)
,
\label{Cexp1}
\end{eqnarray}
%
where $d$ is the distance between $\bx$ and $\bx'$. In our 1D example, $d(x,x') = |x - x'|$. Note~\footnote{
I have used the notation $L' = 2L$ to distinguish our $L'$ from the $L$ that appears in \citet{Tarantola2005}.}.
%The function \verb+covC.m+ inputs $L'$ (assuming that \verb+LFACTOR=2+).
%
$C$~takes in a distance between two points and outputs a value. It can alternatively be written as a function of the two input points, $x$ and $x'$:
%
\begin{eqnarray}
C_{\rm gaus}(x,x') &=&
\sigma^2 \exp \left( - \frac{2 (x-x')^2}{L'^2} \right) 
\\
C_{\rm exp}(x,x') &=&
\sigma^2 \exp \left( - \frac{2 |x-x'|}{L'} \right)
,
\end{eqnarray}
%
or in discrete form
%
\begin{eqnarray}
(C_{\rm gaus})_{kk'} &=& C_{\rm gaus}(x_k,x_{k'}) =
\sigma^2 \exp \left( - \frac{2 (x_k-x_{k'})^2}{L'^2} \right) 
\\
(C_{\rm exp})_{kk'} &=& C_{\rm exp}(x_k,x_{k'}) =
\sigma^2 \exp \left( - \frac{2 |x_k-x_{k'}|}{L'} \right)
,
\end{eqnarray}

\begin{enumerate}
\item What are $C_{\rm gaus}$ and $C_{\rm exp}$ for two points separated by $d = L'$?
\item What are $C_{\rm gaus}$ and $C_{\rm exp}$ for two points separated by $d = L'/2$?
\item What are $C_{\rm gaus}$ and $C_{\rm exp}$ for two points separated by $d = 0$?
\item What values of $\sigma^2$ and $L'$ were used for \refFig{fig:covC2}?
\end{enumerate}
%
Note: Only integers and variables should appear in your answers.

%--------------

\item (0.0) Run the example listed in \verb+covC.m+ and make sure you understand what the input parameters are.

%--------------

\item (0.5) Use \verb+covC.m+ to find a covariance function, $C(d)$, that reasonably fits the scatterplot of $(\bC_{1000})_{kk'}$ versus $D_{kk'}$ from Problem 3-\ref{prob:Cd}.
%
\begin{enumerate}
\item List your values of the parameters that describe $C(d)$.
\item Include a plot with $C(d)$ superimposed on the scatterplot of $(\bC_{1000})_{kk'}$ versus $D_{kk'}$.
\item Let $\bC$ be the covariance matrix corresponding to $C(d)$. \\
What are the diagonal entries of $\bC$ and why?
\item Include a plot of $\bC$ (use \verb+imagesc+).
\end{enumerate}

\label{prob:covC}

\end{enumerate}

%------------------------

%\pagebreak
\subsection*{Problem 4 (3.0). Generating samples from a prescribed covariance}

\citet[][p.~45]{Tarantola2005}:
%
\begin{quote}
\ldots a large enough number of realizations completely characterizes the [Gaussian random] field\ldots Displaying the mean of the Gaussian random field and plotting the covariance is {\em not} an alternative to displaying a certain number of realizations, because the mean and covariance do not relate in an intuitive way to the realizations.
\end{quote}

\noindent
In Problem 3, you used a set of 1000 samples and computed a mean, $\bmu_{1000}$, and a covariance matrix, $\bC_{1000}$. You used $\bC_{1000}$ to estimate a covariance function, $C(d)$, with corresponding covariance matrix $\bC$. Here you will use $\bmu_{1000}$ and $\bC$ (not $\bC_{1000}$) to generate a set of samples that (hopefully) resembles the original samples.

\begin{enumerate}
\item (1.5) 

\begin{enumerate}
\item (1.0) Generate 2000 samples of $\bC$, and save these as a set of $\bem_{\rm C}$ (each $\bem_{\rm C}$ is still $M \times 1$). {\bf Include the pertinent lines of your code.}

Hints:
%
\begin{itemize}
\item \verb+A = chol(C,'lower');+
\item If $\bx = \bA\bw$ is a sample of $\bC$, what are $\bA$ and $\bw$?
\end{itemize}

\item (0.4) Add $\bmu_{1000}$ to each $\bem_{\rm C}$, then plot the first 8 samples (as in Problem 3-\ref{samps}). \\
Superimpose $\bmu_{1000}$ in each subplot.

\item (0.1) Do your samples resemble those provided in Problem 3? (yes or no)
\end{enumerate}

%----------

\item (0.5) Consider the samples of the covariance matrix, $\bem_{\rm C}$.

Compute the mean (\verb+mean+), standard deviation (\verb+std+), and norm of each of the 2000 $\bem_{\rm C}$, and show your results in three histogram plots. Do your results check with what you expect?

NOTE: Matlab's \verb+norm+ command will {\em not} be useful here. In calculating the norm, you will need to use a modified covariance matrix, $M\bC$, where $M \times M$ is the dimension of $\bC$. This will ensure that the norm of each $\bem_{\rm C}$ is about~1.

%----------

\item (0.8) Now generate a new $\bC$ using \verb+covC.m+ by making {\em only} one change: change \verb+icov+ to either 1 or 2. Repeat Problem 4-1 using the same set of Gassian random vectors, $\bw_i$, as before. This will allow for a true comparison between samples from the Gaussian or exponential covariance functions.

\begin{enumerate}
\item Generate samples of the new $\bC$, add $\bmu_{10000}$ to each sample. Plot the the first 8 samples.

\item Describe the differences and similarities between the samples from the two different distributions.

\end{enumerate}

%----------

\item (0.2) Repeat Problem 4-2 for the set of 2000 samples from the new $\bC$.

\end{enumerate}

%------------------------

%\pagebreak
\subsection*{Problem} \howmuchtime\

%-------------------------------------------------------------
\bibliographystyle{agu08}
\bibliography{carl_abbrev,carl_main,carl_source,carl_him,carl_alaska}
%-------------------------------------------------------------

%-------------------------------------------------------------

% generated in covC_plot.m
\clearpage\pagebreak
\begin{figure}
\centering
\includegraphics[width=15cm]{covC_LFACTOR2.eps}
\caption[]
{{
Covariance functions from {\tt covC.m} characterized by length scale $L'$ and amplitude $\sigma^2$.
See \citet[][Section 5.3.3, p. 113]{Tarantola2005}.
Some reference $e$-folding depths are labeled; for example, the $y$-values of the top line is $y = \sigma^2 e^{-1/2} \approx 9.70$.
The Mat\'ern covariance functions include an additional parameter, $\nu$, that influences the shape: $\nu \rightarrow \infty$ for the Gaussian function (upper left), $\nu = 0.5$ for the exponential function (upper right).
\label{fig:covC2}
}}
\end{figure}

%-------------------------------------------------------------
\end{document}
%-------------------------------------------------------------

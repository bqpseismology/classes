% dvips -t letter hw_sumatraB.dvi -o hw_sumatraB.ps ; ps2pdf hw_sumatraB.ps
\documentclass[11pt,titlepage,fleqn]{article}

\usepackage{amsmath}
\usepackage{amssymb}
\usepackage{latexsym}
\usepackage[round]{natbib}
\usepackage{xspace}
\usepackage{epsfig}
\usepackage{bm}

%--------------------------------------------------------------
%       SPACING COMMANDS (Latex Companion, p. 52)
%--------------------------------------------------------------

\usepackage{setspace}    % double-space or single-space

\renewcommand{\baselinestretch}{1.2}

\textwidth 460pt
\textheight 690pt
\oddsidemargin 0pt
\evensidemargin 0pt

% see Latex Companion, p. 85
\voffset     -50pt
\topmargin     0pt
\headsep      20pt
\headheight   15pt
\headheight    0pt
\footskip     30pt
\hoffset       0pt

\include{NEWCOMMANDS}

\graphicspath{
  {./figures/}
}

%--------------------------------------------------------------
\begin{document}
%-------------------------------------------------------------

\begin{center}

{\large \bf Problem Set 6: Analysis of the 2004 Sumatra-Andaman earthquake

Part 2: Analyzing the effects of rupture complexity and Earth heterogeneity

}

GEOS 626: Applied Seismology, Carl Tape

Assigned: March 4, 2014 --- Due: March 13, 2014

Last compiled: \today

\end{center}

%------------------------

\begin{figure}[h]
\centering
\includegraphics[width=12cm]{sumatra_chen.eps}
\caption[]
{{
Rupture model for the 2004-12-26 \magw{9.2} Sumatra earthquake.
This model, produced by Chen Ji, is a modified version of the one originally presented as Model III in \citet[][Figure~5c]{Ammon2005}.
The color corresponds to the seismological moment associated with each patch (red = large).
Plate boundaries are from \citet{Bird2003}: AU = Australia, IN = India, BU = Burma, SU = Sunda.
}}
\label{fig:sumatra_chen}
\end{figure}

%------------------------

\clearpage\pagebreak
\section*{Overview and Instructions}

The purpose of this problem set is to handle a large data set of seismograms and to extract some useful scientific information about the earthquake source (and Earth structure). As you will see, much of the challenge is in representing the seismic waveforms in a sensible manner; this includes judging which seismograms are ``bad,'' that is, not representative of the ground motion. A key part of this representation is determining what bandpass to use for the seismograms for each scientific question.

\begin{itemize}
\item This problem set utilizes some of scripts in \verb+GEOTOOLS+ for analyzing sets of waveforms. See \verb+doc_startupB.pdf+ to set up \verb+GEOTOOLS+.

%---------

\item The main scripts you will use are in your \verb+seis2014+ directory:
%
\begin{itemize}
\item \verb+sumatra_modes_template.m+
\item \verb+sumatra_hf_template.m+
\item \verb+sumatra_lf_template.m+
\end{itemize}

%---------

\item Background reading:

\begin{itemize}
\item Instrument response and fourier analysis: \citet[][Ch.~6]{SteinWysession}
\item Directivity: \citet[][Section 4.3.2]{SteinWysession} 
\item Sumatra earthquake: \citep{Lay2005,Ammon2005,Park2005,Ni2005,SSteinOkal2007}
\item Modes: \citet[][Section 2.9]{SteinWysession} and \citet[][Ch.~8]{DT}. See also ``Computational details'' in Section 10.5.1 of DT.
\item PDFs of all referenced Sumatra papers are available on the network.

%\begin{verbatim}
%/home/admin/databases/SUMATRA/papers/
%/home/admin/databases/SUMATRA/papers/SCIENCE_2005/
%\end{verbatim}

\end{itemize}

%---------

\item To find the index of a certain station with an array of stations names \verb+sta+, try something like this:
%
\begin{verbatim}
imatch = find(strcmp('SBA',sta)==1)
\end{verbatim}
%
Or try out the commands \verb+wkeep+ and \verb+wcut+.


%---------

\item Tips on picking stations for analysis:
\begin{itemize}
\item Pick time series that have the highest signal-to-noise ratios.
\item In several cases there are multiple seismograms for the same station (and component), \eg at the South Pole:
%
\begin{spacing}{1.0}
\begin{verbatim}
120  BHZ_00  IU   QSPA   nm / sec
121  BHZ_10  IU   QSPA   nm / sec
122  BHZ_20  IU   QSPA   nm / sec
123  BHZ_30  IU   QSPA   nm / sec
\end{verbatim}
\end{spacing}
%
(You can see these lines in \verb+sumatra_hf_template.m+.)
You do not need more than one seismogram for each site. In general the multiple listings represent different seismometers being used or different vertical locations, like at the surface or down a deep borehole.

\item For such a big event as Sumatra, there are many seismograms that are ``clipped'' or distorted due to an erroneous response of the seismometer.

\item Having a uniform distribution (by distance, azimuth, latitude, etc) is often needed for analyses. So if you are picking a subset, be sure you have enough stations to cover the particular variation of interest.
\end{itemize}

\end{itemize}

%------------------------

%\pagebreak
\section*{Problem 1 (4.0). Splitting of normal mode frequencies}

Template script: \verb+sumatra_modes_template.m+. DISCLAIMER: Our spectra are computed from the calibration-applied seismograms; the complete instrument response has not been deconvolved, as we did in the previous homework. However, because we are not computing the relative amplitudes between modes, this should not be an issue.

\begin{enumerate}
\item Run \verb+sumatra_modes_template.m+ and examine the stations that were cut from the analysis. Describe three ``errors'' in these records.

\item Browse \verb+all_sumatra_modes.pdf+ to pick a subset of at least 20 stations for this analysis. The stations should be selected based on high signal-to-noise within the frequency range 0.2 to 1.0~mHz, and, notably, the $_0S_2$ peak.
See also ``Tips on picking stations'' in the instructions above.

\item Set \verb+iload = 0+ in \verb+sumatra_modes_template.m+, then modify \verb+ipick+ (or use \verb+wkeep.m+) and plot your selected stations.

Compute the amplitude of the $_0S_0$ peak for all your stations. List the median value.

\item ``Stacking'' refers to summing similar functions in order to enhance the signal-to-noise ratio. Use the function \verb+w2fstack.m+ to generate a stack of your spectra over the frequency range 0--10~mHz. Include a plot of the stacked spectrum over two ranges: 0--10~mHz and 0.2--1.0~mHz. Qualitatively, describe the meaning of this spectrum (\eg why are there spikes)?

\item Several peaks in the Sumatra spectrum are clearly split; the central peak is the ``degenerate'' frequency ($m=0$), and the split peaks are known as ``singlets'' ($m = -l,\ldots,l$). 

Use \verb+w2fstack.m+ to generate a stack of $_0S_2$, and include this plot. Measure the frequency of the central peak, $_0f_2^0$ ($m=0$), and the spacing between peaks, $\Delta f$. Assuming a linear model, write an expression for the singlet frequency $_0f_2^m$.

\item  Analyze the splitting and amplitude of $_0S_2$ as a function of latitude. Qualitatively, how does the relative sizes of the singlet peaks vary as a function of latitude?
%How does the amplitude of the degenerate peak vary with latitude.

See \citet[][Figure 3]{SSteinOkal2007} for additional background.

\item Repeat the previous problem but for epicentral distance (instead of latitude). Epicentral distance can be extracted from the \verb+GCARC+ header. What pattern might you expect to see? (Hint: think about the nodal lines for this mode.)

\item  The PREM theoretical frequencies, $f_0$, can be found in Table V of \citet{PREM}. With these frequencies as a guide, see if you can identify the peaks for $_0S_0$, $_1S_0$, and $_2S_0$ in the stacked Sumatra spectrum. What is special about these peaks and why?

%Using \verb+w2fstack.m+, select appropriate frequency limits to examine the peaks $_0S_0$, $_1S_0$, $_2S_0$. The PREM theoretical frequencies, $f_0$, can be found in Table V of \citet{PREM}; use $\Delta f = 0.04$~mHz on either side of $f_0$ to pick the plotting limits.

See \citet[][p. 106]{SteinWysession} for background.

\end{enumerate}

%------------------------

\section*{Problem 2 (4.0). Directivity}

\begin{enumerate}
\item (0.0) Examine the template script \verb+sumatra_hf_template.m+ and make sure you understand how to do certain operations for plotting record sections. It would be helpful to check what each parameter in \verb+plotw_rs.m+ does (type \verb+open plotw_rs+).

\item (1.0) We will repeat the analysis of \citet{Ni2005} but using even more simplifications than they did. We will assume the following:
%
\begin{itemize}
\item The Earth is flat.
\item The travel time between the fault and any station is encapsulated with the simple velocity $v = 11$~km/s. This is the mean velocity for stations between $30^\circ < \Delta < 85^\circ$ of the source, assuming Jeffrey-Bullen P travel times and arc distances (not distances along the P wave ray path).
\item All stations are ``far'' from the fault, such that the distance from the station to the starting point is approximately equal to the distance from the station to the stopping point.
\end{itemize}
%
As derived in \citet[][Section 4.3.2]{SteinWysession}, the apparent rupture time as measured on a seismogram is given by
%
\begin{equation}
T_r = L\left(\frac{1}{v_r} - \frac{\cos(\alpha-\alpha_0)}{v}\right)
\label{Tr}
\end{equation}
%
where $v_r$ is the rupture velocity, $v = 11$~km/s is the velocity of the medium, $L$ is the fault length, and $\alpha$ is the azimuth to the station, and $\alpha_0$ is the rupture direction.
%
\begin{enumerate}
\item What are $\alpha$ values for the minimum and maximum values of $T_r$?
\item The range of $T_r$ is given by $T_{\rm max} - T_{\rm min}$.
What is the range, considering variations in $\alpha$ only?
\item What is $\overline{T}_r$, the azimuthal average of $T_r$?
\item Show that, with our assumptions, \refeq{Tr} can be written in terms of only $T_{\rm min}$, $T_{\rm max}$, $\alpha$, and $\alpha_0$.
\end{enumerate}

% let A = mean(Tr) = 500 s
% let B = Trmax - Trmin = 200 s
% B = 2L/v --> L = Bv/2
% A = L/vr --> vr = L/A

\item (2.0) Using the template script \verb+sumatra_hf_template.m+, reproduce the analysis of \citet{Ni2005} to estimate the rupture direction ($\alpha_0$), the rupture length ($L$), the rupture time ($\overline{T}_r$), and the rupture velocity ($v_r$). {\bf You do not need to compute the smoothed envelopes---the bandpassed seismograms are enough.} (Though computing the envelopes is easy to do.) Use at least 10 stations in your analysis; note that the epicentral distances must be $30^\circ < \Delta < 85^\circ$.

Include a plot of $T_r$ vs $\alpha$ for your data \citep[see Figure~S1 of][]{Ni2005}.

Hint: when requesting the waveforms, specify \verb+stasub = deg2km([30 85])+; this will give you a starting subset to choose from.

\item (1.0) Using the template script \verb+sumatra_lf_template.m+, show that the directivity can also be inferred from the relative amplitudes of R1 and R2, the ``minor orbit'' and ``major orbit'' Rayleigh wave arrivals\footnote{See Figure 2.7-3 of \citet{SteinWysession}.}. See Figure S1 of \citet{Ammon2005} as a guide.

Is your rupture direction ($\alpha_0$) different for the high-frequency P wave estimate than it is for the long-period Rayleigh wave estimate. Why might this be the case? Hint: see \refFig{fig:sumatra_chen}.
\end{enumerate}

%------------------------

\section*{Problem 3 (2.0). Miscellaneous}

In this problem, you will find it helpful to use an appropriate input for \verb+stasub+ so that \verb+getwaveform.m+ will return only a subset of stations/waveforms.

\begin{enumerate}
\item Using the template script \verb+sumatra_lf_template.m+, show the maximal displacement (not velocity) of some ``near-source'' stations. See Figure S12 of \citet{Ammon2005} for reference and checking.

Include plots of the seismograms for calibration applied (\verb+iprocess=1+) and instrument deconvolved (\verb+iprocess=2+). How different are the waveform shapes and amplitudes?

(Note: It appears that \citet{Ammon2005} used the label H1 to denote the E channel (nominally east) and H2 to denote the N channel (nominally north). )

\item Using the template script \verb+sumatra_hf_template.m+, reproduce the azimuthal record section of P waves shown in Figure S11 of \citet{Ammon2005}. (Plot with one column, not two.) Order the seismograms starting with the direction of the rupture direction (\verb+azcen+ parameter in \verb+plotw_rs.m+). Qualitatively describe the variations in the P waves as a function of azimuth from the rupture direction.

\item Using the template script \verb+sumatra_hf_template.m+, filter the BHZ time series from 4 to 6 Hz. I suggest using the modes-quality stations you selected in Problem 1, but it might be easier just to load the entire data set and browse the stations. See if you observe any post-rupture signals in these records. (Remember what we saw for CAN with the \trange{10}{30} surface waves.)

\end{enumerate}


%------------------------

%\pagebreak
\section*{Problem}

Approximately how many hours did you spend on this problem set? Feel free to suggest improvements here.

%-------------------------------------------------------------
%\pagebreak
\bibliographystyle{agu08}
\bibliography{preamble,refs_carl,REFERENCES,refs_slab}
%-------------------------------------------------------------

%\clearpage

%\input{/home/carltape/latex/misc/wave_params_insert}

%\begin{figure}
%\begin{center}
%\includegraphics[width=15cm]{modes_love_n5_blank.eps}
%\end{center}
%\caption[]
%{{
%Text.
%}}
%\label{fig:love_eigfun_n0}
%\end{figure}

%-------------------------------------------------------------
\end{document}
%-------------------------------------------------------------

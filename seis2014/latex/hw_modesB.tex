% dvips -t letter hw_modesB.dvi -o hw_modesB.ps ; ps2pdf hw_modesB.ps
\documentclass[11pt,titlepage,fleqn]{article}

\usepackage{amsmath}
\usepackage{amssymb}
\usepackage{latexsym}
\usepackage[round]{natbib}
\usepackage{xspace}
\usepackage{epsfig}
\usepackage{bm}

%--------------------------------------------------------------
%       SPACING COMMANDS (Latex Companion, p. 52)
%--------------------------------------------------------------

\usepackage{setspace}    % double-space or single-space

\renewcommand{\baselinestretch}{1.2}

\textwidth 460pt
\textheight 690pt
\oddsidemargin 0pt
\evensidemargin 0pt

% see Latex Companion, p. 85
\voffset     -50pt
\topmargin     0pt
\headsep      20pt
\headheight   15pt
\headheight    0pt
\footskip     30pt
\hoffset       0pt

\include{carlcommands}

\graphicspath{
  {./figures/}
}

%--------------------------------------------------------------
\begin{document}
%-------------------------------------------------------------

\begin{center}
\begin{spacing}{1.2}
{\large \bf Problem Set 7: Love waves for a layered Earth\footnote{This problem set was designed and written by Charles Ammon, Penn State University. I have made some modifications.}}

GEOS 626: Applied Seismology, Carl Tape

Assigned: March 13, 2014 --- Due: April 1, 2014

Last compiled: \today
\end{spacing}
\end{center}

%------------------------

\section*{Overview}

This problem set is a direct extension of the problem set ``Toroidal modes of a spherically symmetric earth.'' Here we examine mode solutions for Love waves of a simple model of a layer over a halfspace. The layer represents the crust, the halfspace the upper mantle.

\begin{itemize}
\item Background reading:

Section 2.7 of \citet{SteinWysession}

Chapter 7 of \citet{AkiRichardsE2}

\item A table of wave parameters is listed in \refTab{tab:waveparm}. For the flat-layered model, the factors such as degree ($l$) and earth radius ($a$) are not relevant; therefore the final two columns are not relevant to this problem.

\end{itemize}

%------------------------

\section*{Background}

In the modes problem set you solved a system of coupled, first-order differential equations.
%
\begin{equation}
\frac{d}{dr}
\left[ \begin{array}{c} W \\ T \end{array} \right]
=
\left[ \begin{array}{cc}
\frac{1}{r} & \frac{1}{\mu(r)} \\
& \\
\frac{(l+2)(l-1)\mu(r)}{r^2} -\omega^2\rho(r)  & \frac{-3}{r}
\end{array} \right]
\left[ \begin{array}{c} W \\ T \end{array} \right]
\label{ODEs}
\end{equation}
%
The analogous problem for a (flat) layer-over-halfspace structure is
%
\begin{equation}
\frac{d}{dr}
\left[ \begin{array}{c} W \\ T \end{array} \right]
=
\left[ \begin{array}{cc}
0 & \frac{1}{\mu(r)} \\
& \\
k^2 \mu(r)-\omega^2\rho(r) & 0
\end{array} \right]
\left[ \begin{array}{c} W \\ T \end{array} \right]
\label{rODEs}
\end{equation}
%
To solve \refEq{ODEs} the strategy was to specify $l$, then search over frequencies ($\omega$) and find solutions, indexed by $n$, in the form $_n\omega_l$, $_nW_l(r)$, $_nT_l(r)$. {\bf To solve \refEq{rODEs} the strategy is to specify a frequency ($\omega = 2\pi/T$), then search over wavenumber ($k$) to find solutions of the form $k_n$, $W_n(r)$, $T_n(r)$.}

Here the layer represents the crust, and the halfspace represents the upper mantle.

\refFig{fig:dots} shows dispersion plots for the toroidal modes (previous homework) and for Love waves (this homework).

%------------------------

\section*{Problem 1 (10.0)}

\begin{enumerate}

\item (0.0) Write down the two equations represented by \refEq{rODEs} and the boundary conditions; show explicit $r$ dependence and $n$ dependence (\eg $\omega_n$). 

%----------------

\item (1.0) The appearance of overtone branches for \refEq{rODEs} depends on the {\em cut-off frequency of the nth higher mode} \citep[][Eq. 7.8]{AkiRichardsE2}\footnote{This is also listed in a different form in \citet[][p. 92]{SteinWysession}.}
%
\begin{equation}
\omega_{{\rm c}n} = \frac{n\pi \beta_c}{H} \left( 1 - \frac{\beta_c^2}{\beta_m^2}\right)^{-1/2}
\end{equation}
%
where $\beta_c$ is the crustal velocity, $\beta_m$ is the mantle velocity, and $H$ is the thickness of the layer.

Consider a model with 40 km thick crust and the following properties for the crust (subscript~$c$) and upper mantle (subscript~$m$):
%
\begin{eqnarray*}
\rho_c &=& 2800 \;{\rm kg/m^3}
\\
\beta_c &=& 3450 \;{\rm m/s}
\\
\rho_m &=& 3300 \;{\rm kg/m^3}
\\
\beta_m &=& 4600 \;{\rm m/s}
\end{eqnarray*}

\begin{enumerate}
\item What are the values of $\mu_m$ and $\mu_c$ ($\mu = \rho\beta^2$)?

\item List the $f_{{\rm c}n} = \omega_{{\rm c}n}/(2\pi)$ for $n$=1--10.

\item Using Figure 7.3 of \citet{AkiRichardsE2} as a guide (but with $f$ instead of $\omega$ for the $x$-axis), {\bf make a semi-qualitative sketch} of the dispersion diagram with the horizontal axis (frequency $f$) ranging from 0 to 0.33 Hz and the vertical axis (phase speed $c$) ranging from $\beta_c$ to $\beta_m$.

\item What is the phase speed of the longest-period Love waves? \\
Plot a dot on your sketch to represent this case.

\item Given a fixed value of $T = 20$ s, how many solutions do you expect? \\
What about for $T = 6$ s? 
\end{enumerate}

\label{prob:sketch}

%\refFig{fig:love_eigfun_n0}

%----------------

\item (0.5) Sketch/draw the setup for this problem and include the following features and labels: the layer, $r$-axis, $r$-origin, $h$, $\lambda_m$, $\rho_c$, $\mu_c$, $\rho_m$, $\mu_m$.

Place the origin ($r = 0$) within the halfspace and assume that $r$ points upward. To avoid numerical overflow during integration, place the origin within the halfspace, about three times a mantle shear-wavelength below the base of the layer.

%----------------

\pagebreak
\item (3.5) Adapt the three scripts from the modes homework to solve \refEq{rODEs}. With this choice of coordinates, we have the initial conditions
%
\begin{eqnarray}
W(r_b) &=& 1.0
\\
T(r_b) &=& \mu_m \sqrt{k^2 - \omega^2/\beta_m^2}
\end{eqnarray}
%
where $r_b = 0$ because of our choice of origin. Note that $T(r_b)$ will change for each $k$.
%
%\begin{equation}
%v_\beta = \sqrt{k^2 - \omega^2/\beta^2}.
%\end{equation}
%
%Use the values of $\beta = \sqrt{\mu/\rho}$ and $\mu$ appropriate for the halfspace, and note that $T(r_b)$ will change for each $k$.

Notes:
%
\begin{itemize}
\item You will need to adapt \verb+spshell_template.m+ to \verb+spshell_love.m+.

Define the following global variables:
%
\begin{verbatim}
global rvec WT rspan k omega
global cthick mmu mrho crho cmu mbeta
\end{verbatim}

\item One of the key parts of the code involves selecting the range of $k$ when searching for the roots $k_n$. Furthermore, consider the order in which $k_n$ are computed in your search. Use the answers to Problem~\ref{prob:sketch} to as a guide. Use the Matlab command \verb+linspace+ to generate a vector of \verb+numk+ values (try \verb+numk=100+).

\item I have provided \verb+surf_stress_love.m+ as a replacement for \verb+surf_stress.m+. Make sure you understand how this works.

\item You will need to adapt \verb+earthfun_love_template.m+ to return $\rho(r)$ and $\mu(r)$.

\item You will need to adapt \verb+stress_disp_tor.m+ (\refeq{ODEs}) to be \verb+stress_disp_love.m+ (\refeq{rODEs}).

Hint: you will want to define the global variables \verb+omega+ and \verb+k+ (only).

\end{itemize}

\begin{enumerate}

\item (1.5) Using the model described in Problem~\ref{prob:sketch}, compute the displacement and stress eigenfunctions for the period $T = 20$.

Check that your root is $k = 8.53 \times 10^{-5}$~1/m.

Show your code for \verb+stress_disp_love.m+ and \verb+earthfun_love.m+.

\item (1.0) Show a plot of the eigenfunctions; draw a line at the base of the layer.

Note: For plotting purposes only, you may find it simpler to transform your $r$ values into depth values, with $z=0$ at the surface.

\item (1.0) Check the system of equations (\refeq{rODEs}) for this solution (see modes homework solutions). You should get much better agreement if you lower the tolerance levels for \verb+ode45+. (After you do the check, you can go back to using the default tolerance.)
\end{enumerate}

%----------------

\item (1.0) Compute the displacement and stress eigenfunctions (show plots) and values of wavenumber for the fundamental mode, $k_0$ ($n=0$), for periods of 120, 80, 50, 25, 10, and 6 seconds.

%----------------

\item (1.0) Compute all possible solutions for the target period of $T = 3.0$ s.
%
\begin{enumerate}
\item Turn in a plot showing the two eigenfunctions, and list your results in the format of the table below. \\

\begin{spacing}{1.5}
\begin{tabular}{c|c|c|c}
\hline\hline
mode branch & $f_n$, mHz & $k_n$, 1/m & $c_n$, m/s \\ \hline\hline
$n=0$ & \hspace{2cm} & \hspace{2cm} & \hspace{2cm} \\ \hline
$n=1$ & & & \\ \hline
\vdots & & & \\ \hline
\end{tabular}
\end{spacing}

\item What is the relationship between the number of zero crossings of $W_n(r)$ in the layer and $n$?
\item What are the two types of waves that are apparent in each eigenfunction?
%Based on the shapes of the eigenfunctions, would you expect higher-$n$ modes to travel faster or slower?
\end{enumerate}

%----------------

\item (1.0) Loop over a range of frequencies to construct the dispersion plot you sketched in Problem~\ref{prob:sketch}. Use an equally spaced frequency vector ($\sim 50$ values) ranging between periods of 3.0 seconds and 20.0 seconds. Sketch (by hand or in Matlab) vertical bars corresponding to $f_{{\rm c}n}$.

Include an additional plot of phase speed versus period, and sketch (by hand or in Matlab) vertical bars corresponding to $f_{{\rm c}n}$ ($T_{{\rm c}n}$).

%----------------

\item (1.0) We now consider the fundamental mode only.

\begin{enumerate}
\item (0.8) Using an equally spaced frequency vector ($\sim$50 values) ranging between periods of 20.0 seconds and 200.0 seconds, compute three dispersion plots for fundamental-mode Love waves:
%
\begin{itemize}
\item (0.4) phase speed~(km/s) vs period~(s)
\item (0.2) phase speed~(km/s) vs frequency~(Hz)
\item (0.2) angular frequency (rad/s) vs wavenumber~(1/km)
\end{itemize}

\item (0.2) Repeat the test but now use lower tolerance with the ODE solver (\verb+ode45+); see the previous modes problem set for details.
%
\begin{itemize}
\item Include the plots with your solution.
\item Why will numerical errors be particularly problematic for calculating group speed, $U = d\omega/dk$?
\end{itemize}

\end{enumerate}

%----------------

\item (1.0)
%
\begin{enumerate}
\item (0.6) Using your code, reproduce Figure 2.8-2 (p.~96) of \citet{SteinWysession}, including both phase speed and group speed, $U = d\omega/dk$.

Notes:
%
\begin{itemize}
\item It might be useful to specify target $k$ corresponding to equal increments of period (since period is being plotted).
\item For numerical differentiation of $A(x)$ to get $dA/dx$, try \verb+gradient(A,x)+.
\end{itemize}

\item (0.4) Explain how measurements of phase speed and group speed can be used to infer structural properties.
\end{enumerate}

%----------------

\end{enumerate}

%------------------------

%\pagebreak
\section*{Problem}

Approximately how many hours did you spend on this problem set? Feel free to suggest improvements here.

%-------------------------------------------------------------
\bibliographystyle{agu08}
\bibliography{preamble,refs_carl,REFERENCES}
%-------------------------------------------------------------

%\clearpage

\input{/home/carltape/latex/misc/wave_params_insert}

%\begin{figure}
%\begin{center}
%\includegraphics[width=15cm]{modes_love_n5_blank.eps}
%\end{center}
%\caption[]
%{{
%Text.
%}}
%\label{fig:love_eigfun_n0}
%\end{figure}

\clearpage\pagebreak
\begin{figure}
\centering
\begin{tabular}{cc}
(a) & \includegraphics[width=12cm]{modes_disp_fig12b_icase7.eps} \\ \hline
& \\
(b) & \includegraphics[width=12cm]{modes_love_n5_imodel1_C.eps} 
\end{tabular}
\caption[]
{{
Comparison between dispersion plots for toroidal modes of a homogeneous shell (a) and Love waves within a layer over a halfspace (b). Note that in (b), the dots are not discrete modes; one should think of the solution space as a set of curves (\ie connect the colored dots). In (a), only the dots are allowable solutions; the curves are drawn to show the difference branches ($n=0$, $n=1$, etc).
\label{fig:dots}
}}
\end{figure}

%-------------------------------------------------------------
\end{document}
%-------------------------------------------------------------

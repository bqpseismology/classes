% dvips -t letter hw_tomo.dvi -o hw_tomo.ps ; ps2pdf hw_tomo.ps
\documentclass[11pt,titlepage,fleqn]{article}

\usepackage{amsmath}
\usepackage{amssymb}
\usepackage{latexsym}
\usepackage[round]{natbib}
\usepackage{xspace}
\usepackage{epsfig}
\usepackage{bm}

\usepackage{pifont}  % \ding

%--------------------------------------------------------------
%       SPACING COMMANDS (Latex Companion, p. 52)
%--------------------------------------------------------------

\usepackage{setspace}    % double-space or single-space

\renewcommand{\baselinestretch}{1.2}

\textwidth 460pt
\textheight 690pt
\oddsidemargin 0pt
\evensidemargin 0pt

% see Latex Companion, p. 85
\voffset     -50pt
\topmargin     0pt
\headsep      20pt
\headheight   15pt
\headheight    0pt
\footskip     30pt
\hoffset       0pt

\include{NEWCOMMANDS}

\graphicspath{
  {./figures/}
}

%--------------------------------------------------------------
\begin{document}
%-------------------------------------------------------------

\begin{spacing}{1.2}
\begin{center}
{\large \bf Problem Set 9: Seismic tomography using ray theory} \\
GEOS 626: Applied Seismology, Carl Tape \\
Assigned: April 10, 2014 --- Due: April 17, 2014 \\
Last compiled: \today
\end{center}
\end{spacing}

%------------------------

\subsection*{Overview}

\begin{itemize}
\item The objective of this assignment is to design a synthetic seismic tomography experiment and to familiarize yourselves with the fundamental aspects of a tomographic inversion:
%
\begin{itemize}
\item model parameterization
\item construction of the matrix of partial derivatives (of each measurement with respect to each model parameter)
\item solving the least-squares problem using regularization
\end{itemize}

\item The source-receiver geometry for this problem is shown in \refFig{fig:geometry}.

\item This example problem can be thought of as a surface-wave tomography problem, whereby the unknown model is a spatial map of surface wave phase speed and the observations are measurements of surface wave traveltimes.

\item The reading and notes on inverse problems from Homework 8 is relevant.
At the end I have listed some additional references.
\nocite{Tape2007,Menke,Aster,Tarantola2005}

\item 
%The practical aspects of the assignment are entirely based on Matlab (plus its Mapping Toolbox), and I have provided several scripts that you will need.
% in a reference directory available from the Blackboard website.
The data files and Matlab scripts that you need are in the directory \verb+tomo/+. The two template scripts are \verb+compute_Gik_ray_template.m+ and \verb+tomo_template.m+.

\end{itemize}

%------------------------

\subsection*{Problem 1 (5.0). Constructing the matrix of partial derivatives.}

\begin{enumerate}
\item (0.5) {\em Model parameterization}. A tomographic model is typically described in terms of a perturbation in wave speed structure from some reference model, \ie $\delta\ln c = \delta c/c$. This function may be expanded in terms of a set of $M$ basis functions:
%
\begin{equation}
\delta\ln c(\bx)=\sum_{k=1}^M m_k\,B_k(\bx),
\label{dlnv}
\end{equation}
%
where $m_k$, $k=1,\ldots,M$, represent the model coefficients describing the model perturbation.  Here we choose to use spherical spline basis functions \citep{WangDahlen1995spline,Wang1998}.

Run the Matlab script \verb+test_spline_vals.m+ in order to familiarize yourself with spherical spline basis functions. This script will generate a spherical spline basis function of grid order $q$, centered at a randomly selected gridpoint.  Change the value for $q$ and see what happens.  An example of a $q = 8$ basis function is shown in \refFig{fig:basis}b.

\begin{enumerate}
\item What is the value of the $q=8$ spline function that is centered at (longitude, latitude) = $(-117^\circ,\;34^\circ)$, evaluated at the point $(-117.05^\circ,\;34.25^\circ)$?
\end{enumerate}

%-----------------

\item (0.5) {\em Formulate an entry of the partial derivatives matrix}.
%
From the class notes, the ray theory traveltime perturbation for the $i$th source-receiver combination is given by
%
\begin{equation}
\delta T_i = -\int_{\mathrm{ray}_i} c^{-1}\,\delta\ln c\;\rmd s,
\label{ray_tomography}
\end{equation}
%
where $\rmd s$ denotes a segment of the $i$th ray.
Show how this leads to $\bd = \bG\bem$, where the data vector is given by
%
\begin{equation}
\bd = (\delta T_1,\ldots,\delta T_i,\ldots,\delta T_N)^T,
\label{d}
\end{equation}
%
and write the expression for the $G_{ik}$ element of the partial derivatives matrix~$\bG$.

%-----------------

\item (4.0) {\em Compute an entry of the partial derivatives matrix}. \\
Now adapt \verb+compute_Gik_ray_template.m+ to compute the $N \times M$ partial derivatives matrix~$\bG$.
Take into account the following details:
%
\begin{itemize}
\item Assume a homogeneous reference phase speed model with $c_0 = 3500$~m/s.

\item Because the reference model is homogeneous, the ray paths are simply great circles.  Use the built-in Matlab function \verb+gcwaypts+ to compute the ray paths, and use 100 points or so to effectively describe the ray path. (I used 1000 points.) The function \verb+distance+ will compute the great-circle distance (in degrees) between two points on a sphere.

\item Use the source--receiver ordering (see also \refFig{fig:geometry}b) given by

\begin{verbatim}
i = (isrc-1)*nrec + irec
\end{verbatim}

Here index $i$ is the row index (or measurement index), \verb+isrc+ is the source index, and \verb+irec+ is the receiver index.
For example, row $i=3293$ of $\bG$ (and $\bd$) corresponds to the pairing of source 25 with receiver 125.
This indexing is shown in \verb+compute_Gik_ray_template.m+.

\item Use $M=286$ spherical spline basis functions of order $q=8$; the center-points are in the data file \verb+con_lonlat_q08.dat+.

\item When properly coded, it should take $<$0.1~s to compute a row of $\bG$ and therefore about 3 minutes to compute the entire matrix. Make sure that your code is reasonably fast (though getting the right answer is most important).

\end{itemize}

\begin{enumerate}
\item What are the units for $d_i$, $m_k$, and $G_{ik}$?

\item Demonstrate that the value of $G_{ik}$ with $i=126$ and $k=204$ is $-10.3747$.

\item What does each row of $\bG$ correspond to?

\item What does each column of $\bG$ correspond to?
\end{enumerate}

%-----------------

\item (0.0) Save the partial derivatives matrix
%
\begin{verbatim}
save('Gik','Gik')    % will save the file Gik.mat with the one variable Gik
\end{verbatim}
%
so that you can load it for the next problem.
%
\begin{verbatim}
load Gik             % will load the file Gik.mat with the one variable Gik
\end{verbatim}

\end{enumerate}

%------------------------

%\pagebreak
\subsection*{Problem 2 (5.0). Solving for the unknown wavespeed structure.}

At this point, you should have saved your partial derivatives matrix. Now open the template file \verb+tomo_template.m+. In order to do the inverse problem, you will need the data vector of traveltime measurements (\refeq{d}). Load the column vector \verb+measure_vec.dat+ into Matlab, and check that it has dimension $3300 \times 1$.

\begin{enumerate}
\item (0.5) Write the symbolic expression for the solution $\bem$ to the least squares problem $\bG\bem=\bd$ in the case where $\bG^T\bG$ is full rank.  List the dimensions of $\bem$, $\bd$, $\bG$, $\bG^T\bG$, and $\bG^T\bd$.

\item (0.5) Compute $\bem$ using your formula.  What do you get, and why is this the case?

\item (1.5) If $\bG^T\bG$ is not full rank, then a damping parameter $\lambda$ may be introduced to stabilize (or ``regularize'') the inversion.  One possible solution to the problem is:
%
\begin{equation}
\bem_{\lambda} = \left(\bG^T\bG+\lambda^2\bI\right)^{-1}\bG^T\bd.
\end{equation}
%
\begin{enumerate}
\item Compute $\bem_{\lambda}$ for a range of $\lambda$ values: for example, try

\verb+lamvec = 10.^linspace(log10(minlam), log10(maxlam), numlam);+

with $\lambda_{\rm min} = 0.1$ and $\lambda_{\rm max} = 100$.

\item Using a \verb+loglog+ plot (or transform your values from $x$ to $\log x$), plot the misfit norm versus model norm. The misfit norm is the sum of the squares of the residuals, where the $i$the residual is $(\bG\bem-\bd)_i$. The model norm is the sum of the squares of the model parameters $m_k$.
\end{enumerate}

\item (1.0) Plot some of these model vectors by expanding them in the spherical spline basis functions, use the same color scale for each plot\footnote{If you want the {\tt seis} color scale from GMT, then type {\tt colormap(seis)} after each figure is done.}, and also include the locations of the sources in the plots. I have pre-computed the necessary matrix, $\bB$, in \verb+tomo_template.m+, such that the function value $\delta\ln c(\phi_i, \theta_i)$ is computed by the dot product $\bB(i,:) \cdot \bem$. This operation can be though of as $\bc_{\lambda} = \bB \bem_{\lambda}$, where $\bc_{\lambda} = \delta\ln c(\phi_i, \theta_i)$ is a vector of phase speed perturbations. (The simplest way to plot might be with the \verb+scatterplot+ command.) Comment on any trends that you see in phase speed maps that you produce.

Use the color scale \verb+caxis([-0.05 0.05])+ for all plots.

\pagebreak
\item (1.0)
%
\begin{enumerate}
\item (0.7) Perform an inversion using only the first five events; this requires using a reduced form of your original partial derivatives matrix. Use the same damping parameters as before, and show and discuss your results.

\item (0.3) Repeat using only the first event.
\end{enumerate}

\item (0.5) What are two ways to stabilize the inverse problem, given total control over the experimental design?  (Hint: How can we construct a matrix $\bG^T\bG$ with fewer zeros?)

\end{enumerate}

%------------------------

\subsection*{Problem.}

Approximately how many hours did you spend on this problem set? Feel free to suggest improvements here.

%-------------------------------------------------------------
%\pagebreak
\bibliographystyle{agu08}
\bibliography{preamble,refs_carl,REFERENCES}
%-------------------------------------------------------------

%\vspace{1cm}
\begin{figure}[h]
%\begin{center}
\centering
\includegraphics[width=17cm]{socal_03b.eps}
\caption[Source--receiver geometry for southern California]
{{
{\em Left}: Example phase speed map and faults for Southern California. The phase speed map is for Rayleigh waves with periods of 20~seconds.
{\em Right}: Source--receiver geometry for the problem in this homework. The \ding{73} symbols denote the locations of 25 earthquakes; the $\circ$ symbols denote the locations of 132 broadband receivers in the Southern California Seismic Network.
\label{fig:geometry}
}}
%\end{center}
\end{figure}

\begin{figure}[p]
\hspace{-1.5cm}
\includegraphics[width=19cm]{talk_Gik03_5_ray_hw.eps}
\caption[Basis function]
{{
Example computation for an element, $G_{ik}$, of the partial derivatives matrix $\bG$, using rays. The row index $i$ is the source-receiver combination, the column index $k$ is the basis function index. The source is denoted by the \ding{73}, the receiver is denoted by the $\triangle$, and the $\circ$ shows the center-point of the spherical spline in (b).
(a)~Ray path for event number~1 and receiver number~126, corresponding to the $i=126$ index of the $N=3300$ ray paths.
(b)~$B_{203}(\bx)$, the spherical spline basis function for index $k=203$. Also shown are the center-points of the $M=286$ spherical splines.
(c)~Spline $B_{203}$ evaluated along the ray path. The value of the phase speed for the reference model is constant, so $G_{ik} = (-1/c) \int_{{\rm ray}_i} B_k\,\rmd s$. In this example, $G_{ik} = -1/(3500\,{\rm m\,s^{-1}})\;(2.45 \times 10^4\,{\rm m})  = -6.98$~s. 
\label{fig:basis}
}}
\end{figure}

%-------------------------------------------------------------
\end{document}
%-------------------------------------------------------------

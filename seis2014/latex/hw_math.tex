% dvips -t letter hw_math.dvi -o hw_math.ps ; ps2pdf hw_math.ps
\documentclass[11pt,titlepage,fleqn]{article}

\usepackage{amsmath}
\usepackage{amssymb}
\usepackage{latexsym}
\usepackage[round]{natbib}
\usepackage{xspace}
\usepackage{epsfig}
\usepackage{bm}

%--------------------------------------------------------------
%       SPACING COMMANDS (Latex Companion, p. 52)
%--------------------------------------------------------------

\usepackage{setspace}    % double-space or single-space

\renewcommand{\baselinestretch}{1.1}

\textwidth 460pt
\textheight 690pt
\oddsidemargin 0pt
\evensidemargin 0pt

% see Latex Companion, p. 85
\voffset     -50pt
\topmargin     0pt
\headsep      20pt
\headheight   15pt
\headheight    0pt
\footskip     30pt
\hoffset       0pt

\include{carlcommands}

\graphicspath{
  {./figures/}
}

%--------------------------------------------------------------
\begin{document}
%-------------------------------------------------------------

\begin{center}

{\large \bf Problem Set 2: Linear algebra and vectors}

GEOS 626: Applied Seismology, Carl Tape

Assigned: January 23, 2013 --- Due: January 30, 2013

Last compiled: \today

\end{center}

%------------------------

\subsection*{Overview}

The purpose of this problem set is to review a few mathematical concepts that are important to seismology. These concepts will be useful in following the class reading \citep{SteinWysession,ShearerE2} and for future homework assignments.

\subsection*{Problem 1 (5.0). Matrix decompositions}

This problem must be done by hand, unless specified. ``By hand'' means showing your steps and not using decimal notation; for example, write the expression $(4 +\sqrt{3})/5$ rather than 1.1464.
Be as clear as possible in explaining what equations you are using.
%You are welcome to check your answers using Matlab.

\medskip\noindent
The matrix for this problem is
%
\begin{equation*}
\bA =  \left[ \begin{array}{rrr}
    -1  &   0  &  -4 \\
     0  &   2  &   1 \\
     1  &   1  &   4 \\
\end{array} \right]
\end{equation*}

%---------------

\begin{enumerate}
\item (0.2) {\bf Determinant.}

Calculate the determinant of $\bA$.

\item (0.5) {\bf Matrix inverse.}

\begin{enumerate}
\item How do we know that the inverse of $\bA$ must exist?
\item Calculate $\bA^{-1}$.
\end{enumerate}

\item (1.0) {\bf Eigenvalues.}

Calculate the eigenvalues of $\bA$, assuming that one of the eigenvalues is $1$.

\item (1.8) {\bf Eigenvectors.}

\begin{enumerate}
\item (1.3) Calculate an eigenbasis of $\bA$ (by hand). Show all row-reduction steps by hand.

\item (0.2) Use Matlab to check that $\bA\bU = \bU\bD$, where $\bU$ is your eigenbasis and $\bD$ is your diagonal matrix with eigenvalues on the diagonal. (Note: I am not asking to use Matlab's \verb+eig+ command.)

\item (0.2) Using Matlab, normalize the columns of $\bU$ and again check that $\bA\bU = \bU\bD$ with your new $\bU$.

\item (0.1) In words, what is the relationship between matrices $\bA$ and $\bD$?
\end{enumerate}

\pagebreak
\item (1.5) {\bf Orthogonalization.}

\begin{enumerate}
\item (1.4) Use the Gram-Schmidt procedure to derive matrices $\bQ$ and $\bR$ of the decomposition of $\bA = \bQ\bR$. See \refFigii{fig:proj}{fig:ortho} as a guide.
%\footnote{If you use Aster p.~301, note that his $\bw_k$ are {\em not} normalized. So his $\bw_k$ are my $\bh_k$, and my $\bw_k = \bh_k/\|\bh_k\|$. See \refFig{fig:ortho}.}

\item (0.1) Using Matlab, check that $\bA = \bQ\bR$ with your $\bQ$ and $\bR$. Also, compare your results Matlab's factorization in \verb+[Q,R] = qr(A)+.
\end{enumerate}

\end{enumerate}

%------------------------

\subsection*{Problem 2 (3.0). Matrix operations}

\begin{enumerate}

\item (0.3) Determine whether the following three vectors are linearly dependent:
%
\begin{eqnarray*}
\left[ \begin{array}{r} 1 \\ -2 \\ 4  \end{array} \right],
\left[ \begin{array}{r} 0 \\ 1 \\ 1 \end{array} \right],
\left[ \begin{array}{r} 1 \\ 4 \\ -2 \end{array} \right]
\end{eqnarray*}

%----------------

\item (0.3) Find the matrix, $\bA$, ($T(\bx) = \bA\bx$) of the following linear transformations:
%
\begin{enumerate}
\item $\mathbb{R}^2 \rightarrow \mathbb{R}^2$ reflection about $x = 0$.
\item $\mathbb{R}^2 \rightarrow \mathbb{R}^2$ reflection about $y = 0$.
\item $\mathbb{R}^3 \rightarrow \mathbb{R}^3$ reflection through $xy$-plane.
\end{enumerate}
%
You are free to use any output from Matlab.

Hint: What does $\bA$ do to the input vector $(x,y)$?

%----------------

\item (0.3)
%You can use Matlab for any tedious steps.

\begin{enumerate}
\item Find the matrix, $\bA$, ($T(\bx) = \bA\bx$) of the linear transformation described by
%
\begin{equation*}
T \left( \left[ \begin{array}{r} 1 \\ 0 \\ 0  \end{array} \right] \right) = \left[ \begin{array}{r} 1 \\ 2 \end{array} \right],
\hspace{0.5cm}
T \left( \left[ \begin{array}{r} 0 \\ 1 \\ 0  \end{array} \right] \right) = \left[ \begin{array}{r} 2 \\ 4 \end{array} \right],
\hspace{0.5cm}
T \left( \left[ \begin{array}{r} 0 \\ 0 \\ 1  \end{array} \right] \right) = \left[ \begin{array}{r} 3 \\ 6 \end{array} \right]
\end{equation*}

\item Find the matrix, $\bA$, ($T(\bx) = \bA\bx$) of the linear transformation described by
%
\begin{equation*}
T \left( \left[ \begin{array}{r} 1 \\ 0 \\ 0  \end{array} \right] \right) = \left[ \begin{array}{r} 1 \\ 1 \\ 0 \end{array} \right],
\hspace{0.5cm}
T \left( \left[ \begin{array}{r} 2 \\ 1 \\ 0  \end{array} \right] \right) = \left[ \begin{array}{r} 1 \\ 0 \\ -1 \end{array} \right],
\hspace{0.5cm}
T \left( \left[ \begin{array}{r} 3 \\ 2 \\ 1  \end{array} \right] \right) = \left[ \begin{array}{r} 0 \\ 1 \\ 1 \end{array} \right]
\end{equation*}

\end{enumerate}

%----------------

\item (0.8) Compute the angle between $\bu = (1,1,1)^T$ and $\bv = (1,2,3)^T$, then compute the area of the parallelogram defined by these vectors using three different methods:
%
\begin{enumerate}
\item Use the height vector, $\bh = \bv - \proj_{\ssL} \bv$, where $\ssL$ is the line in the direction of $\bu$ (\refFig{fig:proj}).
%Sketch the various vectors and the parallelogram.
\item Use the cross product formula, $\bu \times \bv$.
\item Compute $\sqrt{\det(\bA^T \bA)}$, where $\bA = [\;\bu \; \bv\;]$.
\end{enumerate}

%----------------

\item (0.3) \citep[][p.~473, P10]{SteinWysession}.

Prove that the magnitude of a vector is preserved by an orthogonal transformation

Hint: Let $A\bu = \bv$ and shows that $\|\bu\| = \|\bv\|$.

%----------------

\item (0.6) The standard construction of a least squares problem is often given by
%
\begin{equation}
\bG\bdelta\bem = \bd,
\label{lsq}
\end{equation}
%
which has solution\footnote{Note that $\bdelta\bem$ is a single vector.}
%
\begin{equation}
\bdelta\bem = \left( \bG^T\bG \right)^{-1} \bG^T \bd.
\label{lsq_sol}
\end{equation}
%
This formulation in \refEq{lsq} is equivalent to minimizing the function
%
\begin{equation}
F_{\rm lsq}(\bem+\bdelta\bem)
 =  \hlf \sum_{i=1}^N \left(\;\left[ \bG\bdelta\bem \right]_i - d_i\;\right)^2
= \hlf \left\| \bG\bdelta\bem - \bd \right\|_{L2}^2
= \hlf \left( \bG\bdelta\bem - \bd \right)^T \left( \bG\bdelta\bem - \bd \right)
\end{equation}
%
Show that 
%
\begin{eqnarray}
\hlf \left( \bG\bdelta\bem - \bd \right)^T \left( \bG\bdelta\bem - \bd \right)
&=& \hlf \bdelta\bem^T\bA\bdelta\bem + \bb^T\bdelta\bem + \bc
\label{lsq1}
\end{eqnarray}
%
where
%
\begin{equation}
\bA = \bG^T\bG
\,,\hspace{20pt}
\bb = -\bG^T\bd
\,,\hspace{20pt}
\bc = \hlf\bd^T\bd
\,.
\end{equation}

%----------------

\item (0.4) Find all $\bx$ such that $\bA\bx = \bb$, where $\bb = (1, 1, 0)^T$ and
%
\begin{eqnarray*}
\bA =  \left[ \begin{array}{rrrr}
     0  &   0  &   1 & -4 \\
     1  &   -2  &   0 & -3 \\
     -1  &   2  &   1 & -1 \\
\end{array} \right].
\end{eqnarray*}
%
Express the solution as a sum of three vectors.

Matlab is permissible.

\end{enumerate}

%------------------------

\subsection*{Problem 3 (1.0). Div, grad, curl, and all that}

\citep[][p.~473, P13]{SteinWysession} \\
For the vector field $\bu(x,y,z) = \bxh(3x^2y^2 + z) + \byh(2x^3y+2y) + \bzh(x)$, find:
%
\begin{enumerate}
\item (0.4) $\bdel \cdot \bu$
\item (0.2) $\bdel \times \bu$
\item (0.2) $\nabla^2\bu$
\item (0.2) A scalar field $\phi(x,y,z)$ such that $\bu=\bdel\phi$
\end{enumerate}

%------------------------

%\pagebreak
\subsection*{Problem 4 (1.0). Taylor series}

\begin{enumerate}
\item (0.5) Derive the Taylor series expansion of $f(x) = \ln(x)$ about the point $x = x_0$. List terms up to the third-order.

\item (1.0) Let $x_0 = 3.5$. Plot the zeroth-, first-, second-, and third-order terms on a single plot. Plot the zeroth-, first-, second-, and third-order Taylor series on a single plot. (The $n$th-order Taylor series is the sum up to the $n$th-order term.) In all of the plots use $x$ limits from $-1$ to 10.

Normalized residuals can be expressed as $\ln[ f(x) / g_k(x) ]$ where $f(x) = \ln(x)$ and $g_k(x)$ is the $k$th-order Taylor series approximation. What is the approximate range of the normalized residual over the range $x = [3,4]$?

\end{enumerate}

%------------------------

%\pagebreak
\subsection*{Problem}

Approximately how many hours did you spend on this problem set? Feel free to suggest improvements here.

%-------------------------------------------------------------
\bibliographystyle{agu08}
\bibliography{preamble,refs_carl,REFERENCES,refs_alaska,refs_source}
%-------------------------------------------------------------

\clearpage\pagebreak
\input{matrix_fun_figs}

%-------------------------------------------------------------
\end{document}
%-------------------------------------------------------------

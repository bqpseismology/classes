% dvips -t letter hw_matlab.dvi -o hw_matlab.ps ; ps2pdf hw_matlab.ps
\documentclass[11pt,titlepage,fleqn]{article}

\usepackage{amsmath}
\usepackage{amssymb}
\usepackage{latexsym}
\usepackage[round]{natbib}
\usepackage{xspace}
\usepackage{epsfig}
\usepackage{bm}

%--------------------------------------------------------------
%       SPACING COMMANDS (Latex Companion, p. 52)
%--------------------------------------------------------------

\usepackage{setspace}    % double-space or single-space

\renewcommand{\baselinestretch}{1.1}

\textwidth 460pt
\textheight 690pt
\oddsidemargin 0pt
\evensidemargin 0pt

% see Latex Companion, p. 85
\voffset     -50pt
\topmargin     0pt
\headsep      20pt
\headheight   15pt
\headheight    0pt
\footskip     30pt
\hoffset       0pt

\include{NEWCOMMANDS}

\graphicspath{
  {./figures/}
}

%--------------------------------------------------------------
\begin{document}
%-------------------------------------------------------------

\begin{center}

{\large \bf Problem Set 1: 

Introduction to Matlab, featuring the frequency-magnitude relation}

GEOS 626: Applied Seismology, Carl Tape

Assigned: January 16, 2014 --- Due: January 23, 2014

\end{center}

%------------------------

\subsection*{Overview and instructions}

\begin{itemize}

\item The purposes of this problem set are to practice using Matlab and to think critically about histograms. (If you are new to Matlab, let me know.)

\item See the handout \verb+doc_startup.pdf+ to get set up on the Linux network. This includes copying the template file to make \verb+hw_gr.m+.
%to access \verb+GEOTOOLS+.

\item {\bf Possible source of confusion:} Following standard mathematical notation, I will use `$\log$' to represent the base-10 logarithm ($\log_{10}$) and `$\ln$' to represent the natural logarithm. Note that in Matlab \verb+log10+ is $\log_{10}$ and \verb+log+ is $\ln$.

\item {\bf Matlab tips:}
%Use \verb+linspace+ to generate a sequency of numbers, $x_i$; then $10^{x_i}$ will be uniformly spaced in $\log$ space.
\verb+hw_gr_template.m+ shows an example of how to plot multiple items on a log-scaled plot. It also shows how to produce a postscript (PS) or pdf file from a Matlab file. Make sure you follow the template script before proceeding.

\end{itemize}

%------------------------

%\pagebreak
\subsection*{Problem 1 (10.0). Histograms and earthquake statistics}

This problem addresses the famous Gutenberg-Richter frequency-magnitude relation \citep{GutenbergRichter1944}, which is one of the cornerstone empirical relationships in seismology. It is given by
%
\begin{equation}
\log N = a - b\,M,
\label{GR}
\end{equation}
%
where
%
\begin{itemize}
\item $N$ is the cumulative number of earthquakes having magnitudes larger than $M$ that occur in region $R$ within a particular time $T$
\item $M$ is the earthquake magnitude; we take this to be moment magnitude $\mw$
\item $b$ controls the slope of the seismicity distribution in region $R$ within a particular time $T$; $b \approx 1$ for most earthquake catalogs
\item $a$ indicates the seismic activity in region $R$ within a particular time $T$
\end{itemize}
%
A subtle point is that in practice the magnitudes are {\em binned}, then a line is fit to the histogram. It is helpful to consider the discrete form
%
\begin{equation}
\log N_i = a - b\,M_i,
\label{GRd}
\end{equation}
%
where the index $i$ refers to the magnitude bin.

\pagebreak

%------------------

\begin{enumerate}
\item (0.5) We consider the Global Centroid Moment Tensor (GCMT) catalog (\verb+www.globalcmt.org+), from 01-Jan-1976 to 30-June-2011. Thus $T$ represents the duration of the catalog, and $R$ represents planet Earth.

Run the Matlab program \verb+hw_gr_template.m+ to generate a global map of the catalog.
%
\begin{enumerate}
\item What is the range of depths of events in the catalog?
\item List three regions of the deepest seismicity.
\item What is the range of magnitudes of events in the catalog?
\end{enumerate}

\item (1.0) The {\em incremental distribution} is the number of events per magnitude bin, $M_i$. The {\em cumulative distribution} is the cumulative number of events with $M \ge M_i$; the frequency-magnitude relation is based on the cumulative distribution. A {\em magnitude interval} is a range of magnitude, \eg the magnitude interval $[8.7,9.0]$.

As shown in \verb+hw_gr_template.m+, use the function \verb+seis2GR.m+ to obtain the cumulative and incremental distributions for the $\mw$ values of the GCMT catalog, using a bin width of $\Delta M = 0.1$. Examine the output that appears in the command window.

Note: The function \verb+seis2GR.m+ uses the variables names \verb+Ncum+ for the cumulative numbers ($N_i$), \verb+N+ for the incremental numbers, and \verb+Medges+ for the bin edges $M_i$.

\begin{enumerate}
\item What is the maximum value of the incremental distribution?

What is its magnitude interval?

\item What is the maximum value of the cumulative distribution?

What is its magnitude interval?

\item What is the minimum value of the incremental distribution? 

\item What is the minimum value of the cumulative distribution? 
\end{enumerate}

%----------------

\item (2.5) Now it's time to write some lines of code.

Using the output from \verb+seis2GR.m+ (as shown in \verb+hw_gr_template.m+), plot the cumulative and incremental distributions on the same plot, similar to the plot in \citet[][Figure~4.7-2]{SteinWysession}, but note that your $x$-axis is $\mw$, not $\log M_0$.

{\em Matlab tip}: \verb+hw_gr_template.m+ shows an example of how to plot multiple items with \verb+semilogy+ axes. Another alternative is to transform $N$ into $n = \log N$, then work with~$n$.

\begin{enumerate}
\item Which distribution has more scatter?

\item Find a best-fitting line, $\log N = a - b M$, for the `most linear' section of the $\log$-scaled cumulative distribution. You can simply pick two points and compute the line, or use a command such as \verb+polyfit+ (and \verb+polyval+) to apply a least-squares fit to a set of points. (Do not fit a line to the entire distribution!)

Plot your best-fitting line over the full range of magnitudes and include this plot in your write-up. What are your values for $a$ and $b$? What is the physical meaning of $a$? What is the physical meaning of $b$?

\item Assume that the best-fitting distribution (not the GCMT catalog) is `reality'. Based on the idealized cumulative distribution, what is largest earthquake expected over the duration of the GCMT catalog? 

Hint: Where does $N = 1$ intersect your best-fitting line?

Note: The expected value does not have to agree with what actually occurred within the GCMT catalog.

\item The ``catalog completeness'' \citep[\eg][]{WiemerWyss2000}, $M_c$, represents the smallest magnitude of above which the frequency-magnitude is true for a particular seismicity catalog. What is the catalog completeness for GCMT? List your answer with 0.1 precision. (Provide a brief explanation, but no computation is necessary.)

\end{enumerate}

%----------------

\item (1.0) Instead of analyzing seismicity, let us know analyze seismicity rate by dividing all binned values by the duration of the catalog ($T$).
%
\begin{enumerate}
\item Why might seismicity rate be more useful than seismicity?

\item What is the best-fitting line for the new distribution?

\item What magnitude interval averages $>$100 events per year?
\end{enumerate}

%----------------

\item (1.0) Plot the cumulative and incremental distributions for seismicity rate for bin widths of $\Delta M = 0.05, 0.10, 0.5$, and 1.0. What is the apparent relationship between bin width and the separation between the cumulative and incremental distributions?

%----------------

\item (2.0) Define the bin width as
%
\begin{equation}
\Delta M = M_{i+1} - M_i
\label{dM}
\end{equation}
%
where $i$ increases to the right (the usual convention).

The incremental distribution is given by
%
\begin{equation}
-\Delta N_i = -(N_{i+1} - N_i) = N_i - N_{i+1}.
\end{equation}
%
\begin{enumerate}
\item Using the frequency-magnitude relation (\refeq{GRd}), show that
%
\begin{equation}
\log(-\Delta N_i) = a - \Delta a - b M_i
\end{equation}
%
and list the expression for $\Delta a$.

\item What is the relationship between bin width and the shift in the $y$-intercept?

\item If $b = 1$ and $\Delta M = 0.1$, what is $\Delta a$?
\end{enumerate}

%----------------

\item  (1.5) As suggested in \citet[][p.~274]{SteinWysession}, the incremental distribution is related to the derivative of the cumulative distribution. 
%
\begin{enumerate}
\item Using the analytical formula for $d N/ d M$, derive an expression for $\Delta a$ that is valid for small bin widths, $\Delta M \ll 1$.

Hint: What is the mathematical definition of a derivative?

\item If $b = 1$ and $\Delta M = 0.1$, what is $\Delta a$?

\end{enumerate}

%----------------

\item (0.5) Earlier you determined the catalog completeness, $M_c$.
%
\begin{enumerate}
\item Can the GCMT catalog, $M > M_c$ (see earlier part of this problem for $M_c$), be fit with a single line?
\item Compute an estimate $b$ for $M > 7.5$.
\item What does the different $b$ value imply about large events in the catalog?
\item What is a possible reason for this?
\end{enumerate}

\end{enumerate}

%-------------------------------------------------------------

%\pagebreak
\subsection*{Problem}

Approximately how many hours did you spend on this problem set? Feel free to suggest improvements here.

%-------------------------------------------------------------
%\pagebreak
\bibliographystyle{agu08}
\bibliography{preamble,refs_carl,REFERENCES,refs_alaska,refs_source}
%-------------------------------------------------------------

%-------------------------------------------------------------
\end{document}
%-------------------------------------------------------------
